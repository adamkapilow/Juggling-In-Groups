

\documentclass[12nt]{article}
\pagestyle{myheadings}

%Enter your name and Homework set number
%\title{Math 507 HW 3}
%\author{Adam Kapilow}

%Enter your name and Homework set number. This will be a heading on pages after the first page.
%\markright{Adam Kapilow - HW 8}


\usepackage{amsmath, amssymb, amsthm, amsfonts, graphics, enumitem, tikz-cd, framed, hyperref}
\usepackage[space]{grffile}
\usepackage{animate}
\usepackage{tikz}
\usepackage[export]{adjustbox}
\usepackage{cite}


%The following commands allow us to typeset theorems, problems, definitions, etc. 
\theoremstyle{plain}
\newtheorem{theorem}{Theorem}
\newtheorem*{theorem*}{Theorem}
\newtheorem{lemma}{Lemma}
%\newtheorem{innercustomthm}{Theorem}
%\newenvironment{customthm}[1]
%  {\renewcommand\theinnercustomthm{#1}\innercustomthm}
%  {\endinnercustomthm}
% \newtheorem{innercustomlemma}{Lemma}
%\newenvironment{customlemma}[1]
%  {\renewcommand\theinnercustomlemma{#1}\innercustomlemma}
%  {\endinnercustomlemma}
%\newtheorem{lemma}[theorem]{Lemma}
\newtheorem{corollary}[theorem]{Corollary}
\newtheorem*{definition}{Definition}
\newtheorem{problem}[theorem]{Problem}


\input{/Users/adamkapilow/Documents/Academic_Documents/Math/MyDefns.tex}

\begin{document}


\title{Juggling in Groups}
\author{Adam Kapilow}
\maketitle

\newpage
\tableofcontents
\newpage

\begin{abstract}
Juggling patterns have an intimate relationship with time. The construction of juggling patterns largely means carefully managing each throw's relationship to time, when is a ball thrown, and when is it caught. The study of this careful management of time has lead to a wealth of wonderful research on the mathematics of juggling. There are connections to combinatorics, algebraic geometry, graph theory, Markov chains, topological braids, and more. 

Up to this point, all these explorations have had the same basic setting for time, thinking of juggling patterns as permutations on the integers, where the integers index ticks of a juggler's clock. Here we generalize the notion of periodic juggling patterns to equivariant bijections on a group $G$, letting $G$ play the role of time. In the case $G = \Z^d$, we explore the notion of time moving in multiple dimensions, and a deeper study of the case $G = \Z$ yields results on patterns moving forwards and backwards in time. Generalizations of the average theorem and the correspondence with affine permutations are also included.
\end{abstract}

\section{A review of regular siteswaps}




The only necessary mathematical background is a familiarity with the basics of groups and group actions, and some familiarity with taking quotients by group actions. If you want a more concrete introduction to the mathematization of juggling and some motivation for where this all comes from, there are a number of great resources. Burkhard Polster has written a number of great articles and a \underline{\textcolor{blue}{\href{https://www.qedcat.com/books.html}{book on the subject}}} \cite{polster_book}, with a great survey freely available \underline{\textcolor{blue}{\href{https://www.qedcat.com/articles/juggling_survey.pdf}{online}}}, and a \underline{\textcolor{blue}{\href{https://youtu.be/VsQ-OPIZ5kg}{youtube video}}} summarizing the basics. I myself am quite partial to \underline{\textcolor{blue}{\href{https://youtu.be/38rf9FLhl-8}{the lecture}}} Allen Knutson gave at Cornell though that's a longer time commitment at one hour. I won't be discussing state diagrams though they are very interesting.

Let's recall the results on juggling patterns that we wish to generalize. We'll prove all of them later in a more general context, but we just remark on them here first. Throughout, we will only analyze periodic juggling patterns, and simply will not discuss aperiodic patterns. We recall that juggling patterns can be mathematized by assuming a juggler is making throws evenly spaced in time to the ticks of a metronome. Assigning an integer to every tick of the metronome (including ticks in the past) gives an association between the integers $\Z$ and the beats of a juggling pattern. If we assume that a juggler throws at most one ball on any given beat, then a juggling pattern may be given either by specifying a throw heights function $T : \Z \to \N$, or a juggling function $S : \Z \to \Z$, with the relationship $S(i) = T(i) + i$, so that $S$ records when balls are thrown and caught, and $T$ records how high balls are thrown (the convention is $S(i) = i$ when no ball is thrown on beat $i$, for a throw height $T(i) = i - i = 0$). For example, if $S(3) = 7$, we threw a ball on beat 3 that was next thrown on beat 7, and so it was in the air not being manipulated for $T(3) = 4$ beats of time. We are going to allow the balls to be thrown backwards in time as well, so we'll allow $T : \Z \to \Z$. 

\begin{figure}[ht]
\includegraphics[scale=0.7, center]{531_function}
\caption{A portion of a juggling function, where the labels on the x-axis are the throw heights made at that beat, and the heights of the arches are suggestive of the heights of throws. The three different colored trajectories are the paths of the three balls in this pattern. Note that the average of the throw heights is $\frac{5 + 3 + 1}{3} = 3$, the number of balls in the pattern.}
\label{531_function}
\end{figure}


If we assume that no two balls ever land at the same time (a collision in the pattern), then we get that $S$ is one to one/injective. At any beat where a juggler makes a throw, some ball must have landed at that beat in order to have a ball to throw. Combining this with the convention that $S(i) = i$ on any beat where no ball is thrown yields that $S$ is onto/surjective. Both together means that $S$ is a bijection on the integers. This property is reflected in figure \ref{531_function}, where at every beat there's an arch coming in corresponding to a ball landing, and an arch going out corresponding to a ball being thrown. 

Figure 1 also possesses a kind of symmetry. That is, if you shift this picture three units to the right or left you end up with exactly the same picture (though the colors are not all preserved). That is, this juggling function is periodic with period 3, a property which can be expressed in one of two ways. A juggling function $(S, T)$ is periodic with period $n$ if either of the following two equivalent conditions are met. 

\begin{enumerate}
\item
$S$ is equivariant with respect to the action of $n\Z$ on $\Z$. Algebraically, this means $S(i + nq) = S(i) + nq$ for all integers $i, q$. That is, we have a commutative diagram
\[
\begin{tikzcd}
n\Z \times \Z \ar[d, "+"] \ar[r, "(\id \text{, } S)"] & n\Z \times \Z \ar[d, "+"] \\
\Z \ar[r, "S"] & \Z
\end{tikzcd}
\]
where $+(x, y) = x + y$.
\item
$T$ is constant on the orbits of $n\Z \times \Z \xrightarrow{+} \Z$. Algebraically, this means that $T(i + nq) = T(i)$ for all integers $i, q$. That is, we have a diagram
\[
\begin{tikzcd}
\Z \ar[d, "\pi"] \ar[dr, "T"] \\
\Z/n \ar[r, dashed, "\widetilde{T}"'] & \Z
\end{tikzcd}
\]
\end{enumerate}

We can see that the preceding example meets the second condition, as the throw heights just repeat 5, 3, 1, 5, 3, 1, 5, 3, 1, ... and any throw is the same as the throw 3 beats later. The first condition is also met, as we can check that any throws made 3 beats apart land 3 beats apart. We will generalize this to an appropriate alternate characterization of equivariant maps.


The equivariance yields a diagram,
\[
\begin{tikzcd}
\Z \ar[r, "S"] \ar[d] & \Z \ar[d] \\
\Z/n \ar[r, "\widetilde{S}"] & \Z/n
\end{tikzcd}
\]

and we have that $S$ is bijective if and only if $\widetilde{S}$ is, a result occasionally called the permutation test. We will generalize this to show that an equivariant map is bijective if and only if its descent to the quotient is bijective.

\begin{figure}[ht]
\includegraphics[scale=0.4, center]{531_permutation}
\caption{The permutation associated to the pattern in figure 1. This is saying that any ball that's thrown as a 5 is next thrown as a 1, any ball that's thrown as a 1 is next thrown as a 5, and any ball that's thrown as a 3 is next thrown as a 3, as can be verified by looking at figure 1. Note further that each element is mapped to that element plus the labeled throw value mod 3. Image generated at \textcolor{blue}{\underline{\url{https://jbuckland.com/juggling-graph/}}}.}
\end{figure}


A result of immense practical import to jugglers is the the average theorem. That is, suppose we have a juggling function/throw heights pair $S, T$. If $T(i) \geq 0$ for all $i$, then the number of balls in $S$ (mathematically interpreted as the number of infinite orbits of $S$) is the average of the throw heights. That is, we have that $\# \text{balls} = \frac{\sum \limits_{i = 0}^{n-1} T(i)}{n} $, or in the formulation we'll use it:
\[
n \cdot \# \text{balls} = \sum \limits_{i=0}^{n-1} T(i).
\]

If we allow $T$ to have negative values, then it turns out for any given ball in the pattern, we can unambiguously say whether that ball on the whole moves forward in time or backwards in time, or travels along a finite cycle in time. If we let $F$ denote the number of balls on the whole moving forward, and $B$ denote the number of balls on the whole moving backward, then we get
\[
n \cdot (F - B) = \sum \limits_{i=0}^{n-1} T(i).
\]

Let's get cracking!

\section{The general theory}
Our main method of generalizing the juggling discussion is via the symmetry inherent in diagrams like figure 1 that arise from periodic juggling functions. We will encode this symmetry as a kind of action of a group on the pattern, and so we'll endeavor to reframe much of the above discussion in the language of groups and group actions.

Our first task is to obtain a proper analogue of throw heights and juggling functions, and analyze the translation from one to the other. Throughout, if $A$ and $B$ are two sets we will let $\Hom(A, B)$ denote the set of all functions from $A$ to $B$. If a group $G$ acts on a set $X$, then we have an action map $\sigma : G \times X \to X$. We will write $\sigma(g, x) = g \cdot x$. Then for any set $Y$, we get a map
\[
\Hom(Y, G) \times \Hom(Y, X) \cong \Hom(Y, G \times X) \xrightarrow{\sigma_*} \Hom(Y, X),
\] sending a pair of maps $(T, f)$ to the map $S = \sigma_*(T, f)$ defined by $S(y) = T(y) \cdot f(y)$. This is the philosophy of the Yoneda lemma at work, and also shows that this association is natural in $X, Y$ and $G$. 

Since we have a map out of a product of sets, we can apply currying. That is, if we fix a function $f : Y \to X$, then we get a map $\gamma_f : \Hom(Y, G) \to \Hom(Y, X)$ defined by $\gamma_f(T)(y) = T(y) \cdot f(y)$. In the case $G = X = Y = \Z$ and $f = \id$, we have that $T$ is the throw height, and $S = \gamma_{\id}(T)$ is the juggling function. For general $X, Y, G$ and $f$, we have that $T$ is the analogue of the throw heights, and $S = \gamma_f(T)$ is the analogue of the juggling function. That is, we can always turn throw heights into juggling functions via $\gamma_f$. 

To answer the reverse question, of when can we turn juggling functions into throw heights, we recall a lemma about free actions that we'll use repeatedly.

\begin{lemma}
Let the group $G$ act on the set $X$. Then the following two implications are equivalent.
\begin{enumerate}
\item
For any $g \in G$, if there exists an $x \in X$ with $g \cdot x = x$, then $g = \id$.
\item
For any $g, h \in G$, if there exists an $x \in X$ with $g \cdot x = h \cdot x$, then $g = h$.
\end{enumerate}
If either of these two equivalent implications hold, then we say that the action of $G$ on $X$ is \textbf{free}.
\end{lemma}

\begin{proof}
The second condition implies the first, as if we choose $h = \id$ then $\id \cdot x = x$ by assumption on group actions. To see that the first condition implies the second, suppose $g \cdot x = h \cdot x$. Then we have 
\begin{align*}
g^{-1} \cdot (g \cdot x) &= g^{-1} \cdot (h \cdot x) \\
(g^{-1}g) \cdot x &= (g^{-1}h) \cdot x && \text{(associativity of the action)} \\
\id \cdot x &= (g^{-1}h) \cdot x \\
x &= g^{-1}h \cdot x.
\end{align*}
Since the first condition holds, we get that $g^{-1}h = \id$, and multiplying by $g$ we get $h = g$, as desired.
\end{proof}

Now we determine to what degree we can translate between throw heights and juggling functions.

\begin{lemma}
If the action of $G$ on $X$ is free, then for any $f : Y \to X$ the function $\gamma_f : \Hom(Y, G) \to \Hom(Y, X)$ is injective. If the action is transitive, then map $\gamma_f$ is surjective. These both become if and only if statements in the case $Y = X$ and $f = \id$ (or more generally if $f$ is a bijection).
\end{lemma}

\begin{proof}
First let's suppose that $G \curvearrowright X$ is free, and show that $\gamma_f$ is injective. Let's suppose that $\gamma_f(T) = \gamma_f(T')$ then for all $y \in Y$ we have that 
\begin{align*}
T(y) \cdot f(y) &= T'(y) \cdot f(y) \implies \\
T(y) &= T'(y),
\end{align*}
by the preceding lemma.

Now let's suppose $G \curvearrowright X$ is transitive, and show that $\gamma_f$ is surjective. Let $S : Y \to X$ be any function. We wish to build a function $T : Y \to X$ such that $T(y) \cdot f(y) = S(y)$. By transitivity, there always exists some group element $g$ such that $g \cdot f(y) = S(y)$. Picking some such $g$ for each $y \in Y$ (using the axiom of choice) yields the desired function $T$. 

Now we show the reverse implications hold in the case $Y = X$ and $f = \id$. First let's suppose that $\gamma_{\id}$ is injective. Suppose that we have an equality $g_0 \cdot x_0 = x_0$. Let $T : X \to G$ be the function given by $T(x_0) = g_0$, and $T(x) = \id$ for $x \neq x_0$. Let $e : X \to G$ be the function given by $e(x) = \id$ for all $x \in X$. Then the assumption yields that $\gamma_{\id}(T) = \gamma_{\id}(e)$, so by assumption we have that $T = e$. Thus, we have that
\[
\id = e(x_0) = T(x_0) = g_0.
\]
Thus, the action is free. Now let's suppose that $\gamma_{\id}$ is surjective and show that $G \curvearrowright X$ is transitive. Let $x_1, x_2 \in X$. Let $S : X \to X$ be any function with $S(x_1) = x_2$. Then by assumption there exists a function $T : X \to G$ such that $S = \gamma_{\id}(T)$. Applying both functions to $x_1$, we obtain
\[
x_2 = S(x_1) = \gamma_{\id}(T)(x_1) = T(x_1) \cdot x_1.
\]
Thus, $x_2$ is in the orbit of $x_1$. Since $x_1, x_2$ were arbitrary, the action is transitive.
\end{proof}


Now we turn to periodicity questions. If you're familiar with regular siteswap, you might know that the patterns 423 and 423423 encode the same juggling pattern. That is, a pattern of period 3 is also a pattern of period 6, and is also of period $3k$ for any integer $k$. Any juggling pattern thus has a group of periods, consisting of all possible periods for the pattern, including 0. This is what we seek to generalize first. 

\begin{lemma}
Let the group $G$ act on the sets $X$ and $Y$. Let $S : Y \to X$ be any function. Let $\text{Periods}(S) \subset G$ be the set of all group elements $g$ such that for all $x \in X$ we have
\[
S(g \cdot x) = g \cdot S(x).
\]
Then $\text{Periods}(S)$ is a subgroup of $G$.
\end{lemma}

\begin{proof}
It is entirely possible to show that $\text{Periods}(S)$ is a subgroup by direct computation. We give a different approach, though the reader may feel free to pursue direct computation if they are interested.
Note that \[
S(g \cdot x) = g \cdot S(x) \iff g^{-1} \cdot S(g \cdot x) = S(x).
\]
This motivates the introduction of the right action $\rho : \Hom(Y, X) \times G \to \Hom(Y, X)$, denoted $\rho(S, g) = S * g$ given by
\[
[S * g](y) = g^{-1} \cdot S(g \cdot y).
\]
Once we verify that this is indeed a right action, the above computation yields that $\text{Periods}(S)$ is the stabilizer of $S$ under this action, and so is a subgroup of $G$.

To see that $\rho$ is a right action, first note that $S * \id = S$ is immediate. For associativity, we compute
\begin{align*}
[(S * g) * h](y) &= h^{-1} \cdot (S*g)(h \cdot y) \\
	&= h^{-1} \cdot \Big[ g^{-1} \cdot S\big(g \cdot (h \cdot y) \big) \Big] \\
	&= h^{-1}g^{-1} \cdot S(gh \cdot y) && \text{(associativity of the action)} \\
	&= (gh)^{-1} \cdot S(gh \cdot y) \\
	&= [S * gh](y),
\end{align*}
so $[S * g] * h = S * gh$, and $\rho$ is a right action, as desired. 

%We can also show this by direct computation. Let $g, h \in \text{Periods}(S)$. Then we have 
%\begin{align*}
%S(gh \cdot x) &= S(g \cdot(h \cdot x)) && \text{(associativity)} \\
%	&= g \cdot S(h \cdot x) && \text{(assumption on $g$)} \\
%	&= g \cdot (h \cdot S(x)) && \text{(assumption on $h$)} \\
%	&= gh \cdot S(x),
%\end{align*}
%by associativity, as desired. For inverses, let $g \in \text{Periods}(S)$, and note that
%\begin{align*}
%S(x) &= S(\id \cdot x) \\
%	&= S((g \cdot g^{-1}) \cdot x) \\
%	&= S(g \cdot (g^{-1} \cdot x)) \\
%	&= g \cdot S(g^{-1} \cdot x).
%\end{align*}
%Acting by $g^{-1}$ on both sides yields
%\[
%g^{-1} \cdot S(x) = S(g^{-1} \cdot x).
%\]
%Membership of the identity element is similar but easier. 

\end{proof}

For any subgroup $H \subset G$ and function $S : Y \to X$ of $G$-sets, we say that $S$ is $\mathbf{H}$-\textbf{equivariant} if $H \subset \text{Periods}(S)$, with notation as above.

 
Fix a function $f : Y \to X$, and a pair of functions $S : Y \to X$ and $T : Y \to G$ related by $S(y) = T(y) \cdot f(y)$. We next seek a way of generalizing the periodicity statement. To do this, we will also need $Y$ to be a $G$-set, as the periodicity result included $S$ being equivariant. 

Then we get the following result.
\begin{lemma}\label{equivariance_throw_heights}
Let $\sigma : G \times X \to X$ and $\tau : G \times Y \to Y$ be actions of the group $G$ on the sets $X$ and $Y$. Let $f : Y \to X$, $T : Y \to G$, and $S : Y \to X$ be related by $S(y) = T(y) \cdot f(y)$. Let $H \subset G$ be a subgroup, and suppose $f$ is $H$-equivariant (note that the identity map satisfies this requirement). Let $conj_H : H \times G \to G$ denote the conjugation map $(h, g) \mapsto hgh^{-1}$. If the first diagram below commutes, then the second does as well. \textbf{This implication can be reversed if the action of $G$ on $X$ is free}. 
\[
\begin{tikzcd}
H \times Y \ar[r, "(\id \text{, } T)"] \ar[d, "\tau"{name=L}] & H \times G \ar[d, "conj_H"] \ar[rr, phantom, "\implies"] &&  H \times Y \ar[r, "(\id \text{, } S)"] \ar[d, "\tau"] & H \times X \ar[d, "\sigma"{name=R}] \\
Y \ar[r, "T"] & G && Y \ar[r, "S"] \ar[ll, phantom, "\Leftarrow \text{ if } \sigma \text{ free}"] & X,
\end{tikzcd}
\]
Note that the second diagram expresses the statement that $S$ is $H$-equivariant.
\end{lemma} 

We note that the first diagram is also an equivariance diagram, turning the action of left translation into the conjugation action. Before proving the theorem, we note the following corollary

\begin{corollary}
With the notation and setup of the previous theorem, suppose that the action of $G$ on $X$ is free. Suppose further that $G$ is abelian, or more generally that $H \subset Z(G)$, the center of $G$. Then $S$ is $H$-equivariant if and only if $T(h \cdot y) = T(y)$ for all $h, y$. \label{abelian_equivariance_throw_heights} \end{corollary}

To prove this corollary, just note that \[
conj_H(h, T(y)) = hT(y)h^{-1} = hh^{-1}T(y) = T(y),
\]
and apply the theorem. This situation most closely matches the situation of juggling patterns, where the throw heights of a pattern of period $n$ satisfy \newline $T(i + nq) = T(i)$ for all integers $i, q$. 

\begin{proof}
Let $h \in H$ and $y \in Y$. We compute
\begin{align*}
S(h \cdot y) &= T(h \cdot y) \cdot f(h \cdot y) \\
	&= T(h \cdot y) \cdot h \cdot f(y),
\end{align*}
as $f$ is assumed equivariant. Then $S$ is equivariant if and only if this equals $h \cdot S(y)$. If the first diagram commutes, then it equals 
\begin{align*}
hT(y)h^{-1}h \cdot f(y) &= hT(y) \cdot f(y) \\
	&= h \cdot (T(y) \cdot f(y)) \\
	&= h \cdot S(y),
\end{align*}
and so $S$ is $H$-equivariant, i.e. the second diagram commutes. If the action is free and $S$ is $H$-equivariant, then we get 
\begin{align*}
h \cdot S(y) &= S(h \cdot y) \\
h \cdot (T(y) \cdot f(y)) &= T(h \cdot y) \cdot (h \cdot f(y)) \\
[hT(y)] \cdot f(y) &= [T(h \cdot y)h] \cdot f(y).
\end{align*}
Since the action is free, we get that 
\begin{align*}
hT(y) &= T(h \cdot y) h \\
hT(y)h^{-1} &= T(h \cdot y) \\
conj_H(h, T(y)) &= T(h \cdot y),
\end{align*}
the desired commutativity. 
\end{proof}

Now we explore how this data interacts with group homomorphisms. First suppose we have a group $G$ acting on a set $X$, and $G$ is a subgroup of 

For the work that follows, we recall an important fact about quotient spaces (indeed, this is their defining property) that we will give without proof.
\begin{theorem}
Let the group $G$ act on the set $X$, and let $Z$ be another set. Then for any map $f : X \to Z$ such that $f(g \cdot x) = f(x)$ for all $x \in X, g \in G$, there is a unique map $\widetilde{f} : X/G \to Z$ such that the following diagram commutes
\[
\begin{tikzcd}
X\ar[d, "\pi"] \ar[dr, "f"]  \\
X/G \ar[r, "\exists ! \widetilde{f}"', dashed] & Z
\end{tikzcd}
\]
\end{theorem}


Now we wish to generalize the permutation test. To do that, we first need a lemma that lets us descend all the data of juggling patterns to quotient spaces. 
\begin{lemma}\label{throw_heights_descend}
Leg $G$ be a group, and let $H \subset G$ be a normal subgroup. Let $\pi_G : G \to G/H$ denote the quotient homomorphism. Suppose that $G$ acts on the sets $X$ and $Y$. Suppose that $S, f : Y \to X$ and $T : Y \to G$ are related by $S(y) = T(y) \cdot f(y)$, that $S$ and $f$ are both $H$-equivariant, and that $T(h \cdot y) = hT(y)h^{-1}$ for all $h \in H, y \in Y$ (this is the situation of lemma \ref{equivariance_throw_heights}). Then the actions, juggling function, and throw heights all descend to the quotient. That is, we have diagrams
\[
\begin{tikzcd}
G \times Y \ar[d, "\tau"] \ar[r, "(\pi_G \text{, } \pi_Y)"] & G/H \times Y/H \ar[d, dashed, "\widetilde{\tau}"] \\
Y \ar[r, "\pi_Y"] & Y/H
\end{tikzcd}
\]
\[
\begin{tikzcd}
G \times X \ar[d, "\sigma"] \ar[r, "(\pi_G \text{, } \pi_X)"] & G/H \times X/H \ar[d, dashed, "\widetilde{\sigma}"] \\
X \ar[r, "\pi_X"] & X/H
\end{tikzcd}
\]
That is, the actions descend to the quotient. We also have
\[
\begin{tikzcd}
Y \ar[d, "\pi_Y"] \ar[r, "S"] & X \ar[d, "\pi_X"] \\
Y/H \ar[r, dashed, "\widetilde{S}"] & X/H
\end{tikzcd}
\]
\[
\begin{tikzcd}
Y \ar[d, "\pi_Y"] \ar[r, "f"] & X \ar[d, "\pi_X"] \\
Y/H \ar[r, dashed, "\widetilde{f}"] & X/H
\end{tikzcd}
\]
That is, the juggling function descends to the quotient. We also have
\[
\begin{tikzcd}
Y  \ar[r, "T"] \ar[d, "\pi_Y"] & G \ar[d, "\pi_G"] \\
Y/H \ar[r, dashed, "\widetilde{T}"] & G/H
\end{tikzcd}
\]
related by 
\[
\widetilde{S}(\ol{y}) = \widetilde{T}(\ol{y}) \cdot \widetilde{f}(\ol{y})
\]
for all $\ol{y} \in Y/H$. 

In the case that $G$ is abelian, or just $H \subset Z(G)$, we get that $\widetilde{T}$ lifts to a map $T' : Y/H \to G$ such that 
\[
\begin{tikzcd}
Y  \ar[r, "T"] \ar[d, "\pi_Y"] & G \ar[d, "\pi_G"] \\
Y/H \ar[r, "\widetilde{T}"'] \ar[ur, dashed, "T'"'] & G/H
\end{tikzcd}
\]
commutes.
\end{lemma}

\begin{proof}
Let's first show that the dashed arrow $\widetilde{\tau}$ exists, descending the action. That is, for any  $g \in G$, $h_1, h_2 \in H$, and $y \in Y$ we must show that \[\pi_Y(g \cdot y) = \pi_Y\Big((h_1g) \cdot (h_2 \cdot y)\Big).\] To do this, we note that 
\begin{align*}
(h_1g) \cdot (h_2 \cdot y) &= (h_1gh_2) \cdot y \\
	&= (h_1gh_2g^{-1}g) \cdot y \\
	&= (h_1h_3g) \cdot y && \text{(since $H$ is normal, $gh_2g^{-1} \in H$)} \\
	&= (h_1h_3) \cdot [g \cdot y].
\end{align*}
Thus, $[h_1g] \cdot [h_2 \cdot y]$ and $g \cdot y$ are in the same $H$-orbit, and so yield the same element in $Y/H$, as desired, and so the map $\widetilde{\tau}$ exists making its diagram commute. The exact same reasoning on $X$ shows that $\widetilde{\sigma}$ exists making its diagram commute. 

To show that the juggling function descends, we must show that for any $y \in Y, h \in H$ that $\pi_X(S(h \cdot y)) = \pi_X(S(y))$, that is that these two elements are in the same $H$-orbit on $X$. We compute
\[
\pi_X(S(h \cdot y)) = \pi_X (h \cdot S(y)) = \pi_X(S(y)),
\]
as desired. Thus $\widetilde{S}$ exists making its diagram commute, and the exact same reasoning holds for $f$. 

Now we look at the throw heights. First, in the case that $H \subset Z(G)$, we recall that $T(h \cdot y) = T(y)$ for all $h \in H, y \in Y$. That is, $T$ is constant on the orbits of $H$, so the map $T'$ exists, and we can define $\widetilde{T} = \pi_G \circ T'$. 

In general with $H$ a normal subgroup, we again let $h \in H$ and $y \in Y$. We need to show that $\pi_G(T(h \cdot y)) = \pi_G(T(y))$, that is the elements $T(h \cdot y)$ and $T(y)$ differ by left multiplication by an element of $H$. We compute
\begin{align*}
T(h \cdot y) &= hT(y)h^{-1} \\
	&= h T(y) h^{-1}T(y)^{-1}T(y) \\
	&= h h_2 T(y),
\end{align*}
as normality of $H$ yields $T(y)h^{-1}T(y)^{-1} = h_2 \in H$. 

Finally, we show that $\widetilde{S}(\ol{y}) = \widetilde{T}(\ol{y}) \cdot \widetilde{f}(\ol{y})$. We know that $\ol{y} = \pi_Y(y)$ for some $y \in Y$. Then by commutativity of the juggling function diagram we have
\begin{align*}
\widetilde{S}(\ol{y}) &= \widetilde{S} \circ \pi_Y(y) \\
	&= \pi_X(S(y)) \\
	&= \pi_X(T(y) \cdot f(y)) &&\text{(definition of throw heights)}\\
	&= \pi_G(T(y)) \cdot \pi_X(f(y)) && \text{(descending the action)} \\
	&= [\widetilde{T} \circ \pi_Y(y)] \cdot [\widetilde{f} \circ \pi_Y(y)] && \text{(descending $T$ and $f$)} \\
	&= \widetilde{T}(\ol{y}) \cdot \widetilde{f}(\ol{y}),
\end{align*}
as desired.



\end{proof}

Here is the corresponding result for the permutation test.
\begin{theorem}\label{permutation_test}
Let the group $G$ act on the sets $X, Y$, let $H \subset G$ be a subgroup, and let $S : Y \to X$ be $H$-equivariant. Then we have a diagram
\[
\begin{tikzcd}
Y \ar[r, "S"] \ar[d, "\pi_Y"] & X \ar[d, "\pi_X"]\\
Y/H \ar[r, dashed, "\exists ! \widetilde{S}"] & X/H.
\end{tikzcd}
\]
Furthermore, the function $\widetilde{S}$ is surjective if and only if $S$ is, and injective if $S$ is. This last implication can be reversed if the action is free. 

This construction is functorial. That is, $\widetilde{\id} = \id$ and if $Z$ is another $G$-set, and $A : Z \to Y$ is $H$-equivariant, then the diagram
\[
\begin{tikzcd}
Z \ar[r, "A"] \ar[d] & Y \ar[r, "S"] \ar[d, "\pi_Y"] & X \ar[d, "\pi_X"]\\
Z/H \ar[r, "\widetilde{A}"] \ar[rr, bend right, "\widetilde{S \circ A}"'] & Y/H \ar[r, "\widetilde{S}"] & X/H.
\end{tikzcd}
\]
commutes.
\end{theorem}
Before proving this result, we highlight the following immediate corollary

\begin{corollary}
Let the group $G$ act on the sets $X$ and $Y$, with the action on $X$ free. Let $H \subset G$ be a subgroup, and let $S : Y \to X$ be $H$-equivariant. Then we have a diagram
\[
\begin{tikzcd}
Y \ar[r, "S"] \ar[d, "\pi_Y"] & X \ar[d, "\pi_X"]\\
Y/H \ar[r, dashed, "\exists ! \widetilde{S}"] & X/H.
\end{tikzcd}
\]
and $S$ is bijective if and only if $\widetilde{S}$ is.
\end{corollary}
This is the most direct analogue of the permutation test. 

\begin{proof}
The existence and uniqueness of the diagram is proved in the previous lemma. The functoriality is an immediate consequence of the uniqueness. The overall square commutes when the bottom arrow is $\widetilde{S} \circ \widetilde{A}$, and by uniqueness of arrows making that square commute we get $\widetilde{S} \circ \widetilde{A} = \widetilde{S \circ A}$. The fact that $\sim$ sends bijections to bijections is an immediate consequence of this functoriality, but it doesn't take too much work to get a more refined result.

We first prove that if $S$ is surjective, then so is $\widetilde{S}$. First suppose that $S$ is surjective. Let $\ol{x} \in X/H$. Since $\pi_X$ is surjective, we have that $\ol{x} = \pi_X(x)$ for some $x \in X$. Since $S$ is surjective, $x = S(y)$ for some $y \in Y$. Then we have
\[
\widetilde{S}(\pi_Y(y)) = \pi_X(S(y)) = \pi_X(x) = \ol{x}.
\]
Since $\ol{x}$ was arbitrary, we have that $\widetilde{S}$ is surjective. 

Now suppose that $\widetilde{S}$ is surjective, and let $x \in X$. Then there exists some element $\ol{y} \in Y/H$ such that $\widetilde{S}(\ol{y}) = \pi_X(x)$. Since $\pi_Y$ is surjective, there exists some element $y \in Y$ with $\pi_Y(y) = \ol{y}$. By commutativity of the diagram, we get that $\pi_X(S(y)) = \pi_X(x)$. Thus, we have that there is some $h \in H$ so that 
\[
x = h \cdot S(y) = S(h \cdot y).
\]
Thus, $x$ is in the image of $S$. Since $x$ was arbitrary, we have that $\widetilde{S}$ is surjective.

Now we suppose that $S$ is injective, and show that $\widetilde{S}$ is injective. Indeed, suppose $\widetilde{S}(\ol{y}) = \widetilde{S}(\ol{z})$. Since $\pi_Y$ is surjective, there exist elements $y, z \in Y$ with $\ol{y} = \pi_Y(y)$ and $\ol{z} = \pi_Y(z)$. Then we have
\begin{align*}
\pi_X \circ S(y) &= \widetilde{S} \circ \pi_Y(y) &&\text{(commutativity of the diagram)}\\
	&= \widetilde{S} \circ \pi_Y(z) && \text{(by assumption)} \\
	&= \pi_X \circ S(z).
\end{align*}
Thus, we have that $\pi_X(S(y)) = \pi_X(S(z))$. Thus, there exists an $h \in H$ so that \[
S(z) = h \cdot S(y) = S(h \cdot y).
\]
Since $S$ is injective, we have that $z = h \cdot y$, so that 
\[
\ol{y} = \pi_Y(y) = \pi_Y(z) = \ol{z}.
\] Since $\ol{y}$ and $\ol{z}$ were arbitrary, we have that $\widetilde{S}$ is injective. 

Now let's suppose that the action of $G$ on $X$ is free, and that $\widetilde{S}$ is injective. We must show that $S$ is injective. Indeed, suppose $S(y) = S(z)$ for $y, z \in Y$. Then we have that 
\begin{align*}
\widetilde{S} \circ \pi_Y(y) &= \pi_X \circ S(y) \\
	&= \pi_X \circ S(z) \\
	&= \widetilde{S} \circ \pi_Y(z).
\end{align*}
Since $\widetilde{S}$ is injective, we have that $\pi_Y(y) = \pi_Y(z)$. Thus, we have that $z = h \cdot y$ for some $h \in H$. Then we have that 
\begin{align*}
S(y) &= S(z) && \text{(by assumption)} \\
	&= S(h \cdot y) \\
	&= h \cdot S(y).
\end{align*}
Since the action of $G$ on $X$ is free, we get that $h = \id$, and so $z = \id \cdot y = y$. Since $y$ and $z$ were arbitrary, we have that $S$ is injective, as desired.

\end{proof}

The next two sections will concern the situation where $G/H$ is finite, so we pause to make a brief remark about adding in this assumption. Namely, I claim that for $S : G \to G$ a bijection, there exists some $H \subset G$ with $S$ $H$-equivariant and $G/H$ finite if and only if $G/\text{Periods}(S)$ is finite. Obviously if $G/\text{Periods}(S)$ is finite, then we can pick $H = \text{Periods}(S)$. For the reverse implication, if $G$ is $H$-equivariant with $G/H$ finite, then by definition $H \subset \text{Periods}(S)$, and so we get a diagram
\[
\begin{tikzcd}
G \ar[d] \ar[dr] \\
G/H \ar[r, dashed] & G/\text{Periods}(S).
\end{tikzcd}
\]
The dashed arrow is surjective by a diagram chase, and so $G/\text{Periods}(S)$ is finite as it admits a surjection from a finite set.

\section{The average theorem}

Throughout this section, we will let $G$ be group, acting freely on itself via addition. In this context, throw heights and juggling functions are in one to one correspondence. When working with multiple different juggling functions, we will use the notation $T(S)$ to mean the throw heights function of the juggling function $S$, so expressions like $T(S)(x)$ make sense. 

\begin{theorem}\label{throw_sum_H}
Let $G$ be an abelian group, and $H \subset G$ a subgroup. Let $T : G \to G$ and $S : G \to G$ be related by $S(g) = T(g) + g$, with $S$ an $H$-equivariant bijection. Suppose that for all but finitely many $\ol{x} \in G/H$ we have that $T(\ol{x}) = 0$. Then we have $T : G/H \to G$
\[
\op{Sum}(S) := \sum \limits_{\ol{x} \in G/H} T(\ol{x}) \in H
\]

\end{theorem}
Note that this sum has only finitely many nonzero terms by our assumption on $T$, and that this assumption on $T$ holds automatically whenever $G/H$ is finite\footnote{If you have a notion of integrating functions $G/H \to G$ and $G/H \to H$, then I wonder if you should be able to replace this sum with an integral, proving this result via an appropriate change of variables theorem, though I haven't worked out the details yet.}. Note that this assumption on $T$ is equivalent to saying that $S$ fixes every element besides those in finitely many cosets of $H$. 
 Furthermore, in the case of period $n$ siteswaps, $H = n\Z$, and this says that the sum of the throw heights is an integer multiple of the period, which is part of the statement of the average theorem (the rest is giving meaning to that integer multiple).
\begin{proof}
Let $\pi : G \to G/H$ be the quotient homomorphism. We show that $\pi(\op{Sum}(S)) = 0$. For any $\ol{x} \in G/H$, we first note that since $\widetilde{S}(\ol{x}) = (\pi \circ T(\ol{x})) + \ol{x}$, we have that $\pi \circ T(\ol{x}) = \widetilde{S}(\ol{x}) - \ol{x}$. Let $F \subset G/H$ be those elements with nonzero throw heights. We compute
\begin{align*}
\pi(\op{Sum}(S)) &= \pi \left(\sum \limits_{\ol{x} \in G/H} T(\ol{x}) \right) \\
	&= \sum \limits_{\ol{x} \in G/H} \pi \circ T(\ol{x}) \\
	&= \sum \limits_{\ol{x} \in F} \pi \circ T(\ol{x}) \\
	&= \sum \limits_{\ol{x} \in F} \widetilde{S}(\ol{x}) - \ol{x} \\
	&= \sum \limits_{\ol{x} \in F} \widetilde{S}(\ol{x}) - \sum \limits_{\ol{x} \in F} \ol{x},
\end{align*}

Since $S$ is a bijection, our previous work yields that $\widetilde{S}$ is a bijection, which thus restricts to a bijection on those points that are not fixed by $\widetilde{S}$. Thus, the two sums in this last line are sums of the same terms, but in a different order. Upon subtracting, these sums cancel out, and we get $\pi(\op{Sum}(S)) = 0$, and so $\op{Sum}(S) \in H$, as desired. 

\end{proof}

\begin{corollary}
Under the same assumptions as in the previous theorem, let $\widetilde{S} : G/H \to G/H$ the associated bijection on $G/H$. Let $\ol{\mathcal{O}}$ be an orbit of $\widetilde{S}$. Then \[
\sum \limits_{\ol{g} \in \ol{\mathcal{O}}} T(\ol{g}) \in H
\]
\end{corollary}

\begin{proof}
We can define a new function $\sigma$ on $G$ given by 
\[
\sigma(g) = \begin{cases}
S(g) & \text{if } \pi(g) \in \ol{\mathcal{O}} \\
g & \text{otherwise}. 
\end{cases}
\]
It is immediately verified that $\sigma$ is $H$-equivariant.\footnote{Taking the composition of all these constructed functions over all the orbits of $\widetilde{S}$ yields a decomposition of $S$ as a product of commuting elements, as is the usual case for finite permutations.} Furthermore, its descended function $\widetilde{\sigma} : G/H \to G/H$ is defined by
\[
\widetilde{\sigma}(\ol{g}) = \begin{cases}
\widetilde{S}(\ol{g}) & \text{if } \ol{g} \in \ol{\mathcal{O}} \\
\ol{g} & \text{otherwise},
\end{cases}
\]
which is quickly verified to be a bijection, using properties of orbits. Thus, we have that $\sigma$ is an $H$-equivariant bijection, and so the result above applies to $\sigma$. We compute
\[
T(\sigma)(\ol{g}) = \begin{cases}
T(S)(\ol{g}) & \text{if } \ol{g} \in \ol{\mathcal{O}} \\
0 & \text{otherwise}.
\end{cases}
\]
Since we're replacing some throws of $S$ by zero, we still only have finitely many nonzero throws, and the previous result applies. We compute
\[
\sum \limits_{\ol{g} \in \ol{\mathcal{O}}} T(S)(\ol{g}) = \sum \limits_{\ol{g} \in G/H} T(\sigma)(\ol{g}) \in H,
\]
where the last containment is the previous result applied to $\sigma$.
\end{proof}

So now we have a function $\op{Sum} : \op{Orb}(\widetilde{S}) \to H$ given by summing the throw heights in an orbit. Now we endeavor to interpret the outputs of this function in a manner similar to the average theorem. To do that, we'll first need two lemmas.

\begin{lemma}
Let $G$ be a group and $H \subset G$ a normal subgroup (no finiteness assumptions in this lemma). The map $S \mapsto \widetilde{S}$ is a homomorphism from the set of $H$-equivariant bijections to $\op{Perm}(G/H)$. Furthermore, the map $\pi : G \to G/H$ respects the orbits of $S$, yielding a surjective map $\pi : \op{Orbits}(S) \to \op{Orbits}(\widetilde{S})$ fitting into a diagram
\[
\begin{tikzcd}
G \ar[d] \ar[r] & G/H \ar[d] \\
\op{Orb}(S) \ar[r, "\pi"] & \op{Orb}(\widetilde{S})
\end{tikzcd}
\]
\end{lemma}

\begin{proof}
The first statement is the functoriality of theorem \ref{permutation_test}. The surjectivity on orbits will follow immediately from the commutativity of the diagram and surjectivity of all the other arrows. To show that this diagram exists, we just need to show that if $x, y \in G$ are in the same orbit of $S$, then $\pi(x), \pi(y) \in G/H$ are in the same orbit of $\widetilde{S}$. To see this, note that if $x, y \in G$ are in the same orbit of $S$, then $y = S^{\ell}(x)$ for some integer $\ell$ (not necessarily positive). Then we have
\begin{align*}
\pi(y) &= \pi \circ S^{\ell}(x) \\
	&= \widetilde{S^{\ell}} \circ \pi(x) && \text{(definition of $\sim$)}\\
	&= \widetilde{S}^\ell \circ \pi(x) && \text{(as $\sim$ is a homomorphism)} \\
	&= \widetilde{S}^\ell(\pi(x)),
\end{align*}
so $\pi(x)$ and $\pi(y)$ are in the same orbit, as desired.
\end{proof}

\begin{lemma}
Fix a group $G$ and a subgroup $H$ (no finiteness assumptions). Let $S$ be an $H$-equivariant bijection $S : G \to G$  Then the action of $H$ on $G$ descends to an action of $H$ on $\op{Orb}(S)$. If $H$ is normal, the map $\pi : \op{Orb}(S) \to \op{Orb}(\widetilde{S})$ exists and is invariant with respect to this action, and $H$ acts transitively on the fibers of $\pi$.
\end{lemma}

\begin{proof}
The idea is that the action of $H$ sends orbits of $S$ to orbits of $S$ because the actions of $S$ and $H$ on $G$ commute by the assumed $H$-equivariance of $S$ (which if you like also means that the action of $S$ preserves the orbits of $H$, which is why it descended to $G/H$). More formally, suppose $x, y \in h \cdot \mathcal{O}$ with $\mathcal{O}$ an orbit of $S$ and $h \in H$. Then $h^{-1}y, h^{-1}x \in \mathcal{O}$, so there is an integer $\ell$ such that
\[
h^{-1}y = S^{\ell}(h^{-1}x) = h^{-1}S^{\ell}(x) \implies y = S^{\ell}(x).
\]
Similarly, if $z \in G$ is such that $z = S^{\ell}(x)$ for $x \in h \cdot \mathcal{O}$, then we get
\[
h^{-1}z = h^{-1}S^{\ell}(x) = S^{\ell}(h^{-1}x).
\]
Since $h^{-1}x \in \mathcal{O}$, we get that $h^{-1}z \in \mathcal{O}$, so $z \in h \cdot \mathcal{O}$, and this set is an orbit, as desired.

Now we show that the action of $H$ on orbits is transitive on the fibers of $\pi$. To do this, let $\mathcal{M}, \mathcal{N}$ be orbits with $\pi(\mathcal{M}) = \pi(\mathcal{N})$. Let $m \in \mathcal{M}$. Then $\pi(m) \in \pi(\mathcal{N})$, so $\pi(m) = \pi(n)$ for some $n \in \mathcal{N}$. Thus, we have that $m = h \cdot n$ for some $h \in H$. Thus, we have that $\mathcal{M}$ and $h \cdot \mathcal{N}$ are two orbits with nonempty intersection, and so they must be equal, and the action is transitive on fibers. 

\end{proof}


Now we can state our version of the average theorem. For motivation, we take another look at figure \ref{531_function} again, repeated below for convenience.

\begin{figure}[ht]
\includegraphics[scale=0.6, center]{531_function}
\end{figure}

 We can shift the diagram to the right by the period, 3, and obtain the same overall diagram. However some but not all of the trajectories of the balls are preserved by this shift. For instance, shifting right by 3 units swaps the red and blue balls. However shifting right by 6 units preserves the path of the red and blue balls. This size 6 is the distance between each iteration of the corresponding orbit in $\Z/3$ (if you like, a kind of overall wavelength of the red arches). But that distance traveled is exactly the sum of the throw heights in the $\Z/3$ orbit. That is, the red ball travels a distance of 5 units, then a distance of 1 unit, then repeats. So we have that the stabilizer of the red ball under this action of the period is exactly the sum of the throw heights in that ball's $\Z/3$-orbit. 
 
 The same is true for the black ball, though simpler. There shifting by the period 3 does preserve the black ball, and 3 is the sum of throw heights in that ball's singleton $\Z/3$-orbit. Here is the formal statement.


\begin{theorem}
Let $S$ be an $H$-equivariant bijection on $G$ with $G$ abelian and $H \subset G$ a subgroup. Suppose further that $T(\ol{x}) = 0$ for all but finitely many $\ol{x} \in G/H$. The stabilizer of any orbit in $\pi^{-1}(\ol{\mathcal{O}})$ is the subgroup generated by $\op{Sum}(\ol{\mathcal{O}})$. Thus, the orbit-stabilizer theorem yields isomorphisms of $H$-sets
\[
\op{Orb}(S) \cong \coprod \limits_{\ol{\mathcal{O}} \in \op{Orb}(\widetilde{S})} \pi^{-1}(\ol{\mathcal{O}}) \cong \coprod \limits_{\ol{\mathcal{O}} \in \op{Orb}(\widetilde{S})} \frac{H}{\langle \op{Sum}(\ol{\mathcal{O}})\rangle }.
\]
with projection to $\op{Orb}(\widetilde{S})$ given by sending a point in this disjoint union to the orbit indexing its piece of the disjoint union.
\end{theorem}
Before proving this theorem, we note how it reduces to the standard average theorem
\begin{corollary}
Let $G = \Z$ and $H = n\Z$ in the above theorem. Then for any orbit $\ol{\mathcal{O}} \in \op{Orb}(\widetilde{S})$ with $\op{Sum}(\ol{\mathcal{O}}) \neq 0$ we have
\[
\# \text{balls in } \ol{\mathcal{O}} = |\pi^{-1}(\ol{\mathcal{O}})|^{-1} = \left| \frac{\sum \limits_{\ol{x} \in \ol{\mathcal{O}}} T(\ol{x})}{n} \right|.
\]
When $\op{Sum}(\ol{\mathcal{O}}) = 0$, any two orbits mapping to $\ol{\mathcal{O}}$ differ by a multiple of the period, and each orbit is isomorphic to $\ol{\mathcal{O}}$ under $\pi$.
\end{corollary}

The first part of this corollary follows from noting that $n\Z/nk\Z \cong \Z/k\Z$, which has size $|k|$ when $k \neq 0$, and that $n\Z/0 \cong n\Z$, so the orbits exactly differ by adding multiples of the period, and so the projection $\Z \to \Z/n\Z$ is injective on each orbit, and so bijective onto its image, $\ol{\mathcal{O}}$. 

Now we prove the theorem.

\begin{proof}
Let $\mathcal{O}$ be an orbit of $S$, let $\ol{\mathcal{O}} = \pi(\mathcal{O})$, and let $d = |\ol{\mathcal{O}}|$, which is finite by our assumption on $S$ (the orbits have size one whenever the throw height is zero). Then restricting our attention to $\pi(\mathcal{O})$, the orbit stabilizer theorem yields that $\Z/d \cong \pi(\mathcal{O})$ under any map $i \mapsto \widetilde{S}^i(\ol{x})$ for fixed $\ol{x} \in \pi(\mathcal{O})$. 

Let $x \in \mathcal{O}$, and let $m > 0$ be a positive integer. We compute
\begin{align*}
S^{d}(x) - x &= \sum \limits_{j = 0}^{d - 1} S^{j+1}(x) - S^j(x) \\
	&= \sum \limits_{j = 0}^{d - 1} T(S^j(x)) \\
	&= \sum \limits_{j = 0}^{d - 1} T \circ \pi \circ S^j(x) \\
	&= \sum \limits_{j = 0}^{d - 1} T(\widetilde{S}^j(\ol{x})) \\
	&= \op{Sum}(\ol{\mathcal{O}}),
\end{align*}
The last equality uses $\Z/d \cong \ol{\mathcal{O}}$ again. Rearranging we have 
\[
S^{d}(x) = x + \op{Sum}(\ol{\mathcal{O}})
\]
This shows $\op{Sum}(\ol{\mathcal{O}}) \in \op{Stab}(\mathcal{O})$, and so $\langle \op{Sum}(\ol{\mathcal{O}}) \rangle \subset \op{Stab}(\mathcal{O})$.  

For the reverse containment, suppose $\mathcal{O} = \mathcal{O} + h$ for some $h \in H$. Fix $x \in \mathcal{O}$. Then we have that $x + h \in \mathcal{O}$, so that $x + h = S^{\ell}(x)$ for some integer $\ell$. Applying $\pi$ yields $\ol{x} = \widetilde{S}^{\ell}(\ol{x})$, and so by orbit-stabilizer again we get that $d$ divides $\ell$, say $\ell = md$ where we may suppose WLOG that $m \geq 0$. Thus, we have
\begin{align*}
h &= S^{md}(x) - x \\
	&= \sum \limits_{j=0}^{m - 1} S^{(j+1)d}(x) - S^{jd}(x) \\
	&= \sum \limits_{j=0}^{m-1} S^d(S^{dj}(x)) - S^{dj}(x) \\
	&= \sum \limits_{j=0}^{m-1} \op{Sum}(S) && \text{(by the preceding calculation applied to $S^{dj}(x))$} \\
	&= m \op{Sum}(S),
\end{align*}
and so $h = m\op{Sum}(\ol{\mathcal{O}}) \in \langle \op{Sum}(\ol{\mathcal{O}}) \rangle$. Since $h$ was arbitrary, we have that $\op{Stab}(\mathcal{O}) \subset \langle \op{Sum}(\ol{\mathcal{O}})$, and thus $\op{Stab}(\mathcal{O}) = \langle \op{Sum}(\ol{\mathcal{O}})\rangle$, and the result holds. 

\end{proof}

If you sum across the whole pattern ignoring orbits, you get some cancellation which we elucidate now. Let $S : \Z \to \Z$ be a siteswap of period $n > 0$. Let $\mathcal{O}$ be an orbit of $S$. We say that $\mathcal{O}$ is \textbf{forward-moving} if $\op{Sum}(\pi(\mathcal{O})) > 0$, and we say that $\mathcal{O}$ is \textbf{backward-moving} if $\op{Sum}(\pi(\mathcal{O})) < 0$. We will justify the terminology in a moment. We've shown above that $\mathcal{O}$ is finite if and only if $\op{Sum}(\pi(\mathcal{O})) = 0$. Then we have the following corollary
\begin{corollary}
Let $S : \Z \to \Z$ be a siteswap of period $n > 0$. Let $F$ denote the number of forward-moving orbits, and let $B$ denote the number of backward-moving orbits. Then we have
\[
n(F - B) = \sum \limits_{\ol{x} \in \Z/n} T(\ol{x})
\]
Furthermore, if the throws of $S$ have constant sign, then 
\[
n \cdot \# \text{balls} = \left|\sum \limits_{\ol{x} \in \Z/n} T(\ol{x}) \right|
\]
\end{corollary}

\begin{proof}
The second equality will follow from the first, as the collection of non-singleton orbits (corresponding to balls you actually throw) is entirely made up of either all forward or all backward-moving orbits. 

For the first equality, we have
\begin{align*}
\sum \limits_{\ol{x} \in \Z/n} T(\ol{x}) &= \sum \limits_{\ol{\mathcal{O}} \in \op{Orb}(\widetilde{S})} \sum \limits_{\ol{x} \in \ol{\mathcal{O}}} T(\ol{x})
\end{align*}
For any orbit $\ol{\mathcal{O}}$ with $\op{Sum}(\ol{\mathcal{O}}) = 0$ it contributes nothing to this sum, and no orbit in its fiber contributes to $F - B$. For any orbit with $\op{Sum}(\ol{\mathcal{O}}) \neq 0$, it contributes $\pm n|\pi^{-1}(\ol{\mathcal{O}})|$, with a plus sign if all the orbits are forward orbits, and a minus sign if all the orbits are backwards orbits. Adding up these contributions over all orbits of $\widetilde{S}$ then gives $n(F - B)$, as desired.
\end{proof}

The reason for the choice of nomenclature is given in the following lemma.

\begin{lemma}
Let $\mathcal{O}$ be an orbit of $S$. Then $\mathcal{O}$ is
\begin{enumerate}
\item Forward-moving if and only if $\lim \limits_{\ell \to \infty}S^\ell(x) - x= \infty$ for any $x \in \mathcal{O}$.
\item Backward-moving if and only if $\lim \limits_{\ell \to \infty} S^\ell(x) - x = - \infty$ for any $x \in \mathcal{O}$
\item Finite if and only if $S^\ell(x) - x$ oscillates around finitely many values for any $x \in \mathcal{O}$.
\end{enumerate}

Furthermore, for infinite orbits, the convergence
\[
S^{\ell}(x) - x \to \pm \infty \text{ as } \ell \to \infty
\]
is uniform in $x$. \end{lemma}

\begin{proof}
Let $x \in \mathcal{O}$, and let $|\ol{\mathcal{O}}| = d$. As we've shown previously, we have $S^{md}(x) = x + m\op{Sum}(\ol{\mathcal{O}})$, with $\op{Sum}(\ol{\mathcal{O}}) \in n\Z$. Then for any integer $\ell$ we may write $\ell = md + r$ with $0 \leq r \leq d - 1$ and compute
\begin{align*}
S^{\ell}(x) - x &= S^{md + r}(x) - x \\
	&= S^r(S^{md}(x)) - x \\
	&= S^r(x + m \op{Sum}(\ol{\mathcal{O}})) - x \\
	&=  m \op{Sum}(\ol{\mathcal{O}}) + S^r(x)  - x,
\end{align*}
where the last equality comes from the fact that $\op{Sum}(\ol{\mathcal{O}}) \in n\Z$ and $S^r$ is equivariant with respect to $n\Z$. Then $m \to \infty$ as $\ell \to \infty$. If $\op{Sum}(\ol{\mathcal{O}}) = 0$, this shows that these values oscillate between the finitely many possibilities $S^r(x) - x$ for $0 \leq r \leq d - 1$. If $\op{Sum}(\ol{\mathcal{O}}) > 0$, then this sum tends to infinity as $m \to \infty$, and if $\op{Sum}(\ol{\mathcal{O}}) < 0$ then this sum tends to negative infinity as $m \to \infty$. 

Furthermore, as we've seen before $S^r(x) - x = \sum \limits_{i=0}^{r-1} T(S^i(x))$, a sum of some subset of the throw heights in $\ol{\mathcal{O}}$ (no repeats occur as $0 \leq r < d = |\mathcal{O}|$). There are only finitely many possible such subsets, so the term $S^r(x) - x$ can take on only finitely many possible values, and those values are independent of $x$, yielding the desired uniform convergence.

Since exactly one of the three statements $\mathcal{O}$ is forward moving, backward moving, of finite holds, and they each imply three disjoint limiting possibilities, then we get the reverse implications for free. For example, suppose that $\mathcal{O}$ was an orbit with some $x \in \mathcal{O}$ such that $\lim \limits_{\ell \to \infty} S^\ell(x) - x = \infty$, and suppose towards a contradiction that $\mathcal{O}$ was not forward moving. Then it would be either backward moving or finite, both of which imply a different limiting result then this. So neither of these situations are possible, and thus we must have that $\mathcal{O}$ is forward-moving. The other possibilities work out similarly.

\end{proof}


\newpage

\section{The structure of $\op{Swap}_H(G)$}

Let $G$ be a group, and let $H \subset G$ be a normal subgroup. We define $\op{Swap}_H(G)$ as the set of all $H$-equivariant bijections $S : G \to G$. 

\begin{lemma}
The set $\op{Swap}_H(G)$ is a group under composition.
\end{lemma}

\begin{proof}
Since the set of all bijections $G \to G$ forms a group under composition, and $\op{Swap}_H(G)$ clearly contains the identity element, we just need to show that $\op{Swap}_H(G)$ is closed under composition and taking inverses.

To do this, let $A, B \in \op{Swap}_H(G)$. Let $g \in G$ and $h \in H$. Then we compute
\begin{align*}
A \circ B(hg) &= A(B(hg)) \\
	&= A(hB(g))  && \text{(as $B$ is $H$-equivariant)}\\
	&= hA(B(g)) && \text{(as $A$ is $H$-equivariant).}
\end{align*}
Thus, we have that $A \circ B \in \op{Swap}_H(G)$

For inverses, to show that $A^{-1}(hg) = hA^{-1}(g)$, it is equivalent to show that $hg = A(hA^{-1}(g))$. But we have that
\[
A(hA^{-1}(g)) = hA(A^{-1}(g)) = hg,
\]
as desired.
\end{proof}

If we have an inclusion $\iota : G \hookrightarrow K$, then we get a map $\iota_* : \op{Perm}(G) \to \op{Perm}(K)$ given by
\[
\iota_*(\sigma)(k) = \begin{cases}
k & \text{if } k \not \in \op{im}(\iota) \\
\iota(\sigma(g)) & \text{if } k = \iota(g).
\end{cases}
\]
It may be quickly verified that $\iota_*$ is a homomorphism, and it sends $H$-equivariant permutations to $\iota(H)$-equivariant permutations. Thus, we have a functor $\op{Swap}_{-}(-)$ from the category of pairs $(G, H)$ with $H \subset G$ a subgroup, where a morphism $(G, H) \to (G', H')$ is an inclusion $G \hookrightarrow G'$ sending $H$ isomorphically to $H'$. Note that for the subcategory of Abelian groups, $\iota_*$ preserves those siteswaps satisfying the finiteness condition for the average theorem.


We have two important pieces of data defining elements in $\op{Swap}_H(G)$. First, we have the group $\op{Perm}(G/H)$ of permutations of $G/H$ (just set permutations, not group isomorphisms). We have a projection function $P : \op{Swap}_H(G) \to \op{Perm}(G/H)$ where $P(S)$ is the descent of $S$ to $\widetilde{S} : G/H \to G/H$. As we've shown previously, $P$ is a homomorphism. In fact, 

The second important piece of data is the throw heights function. Indeed, the throw heights function uniquely determines the siteswap, so it's relevant to see what happens to the throw heights under composition. For this, we have the following lemma:

\begin{lemma}
For any $A, B \in \op{Swap}_H(G)$ and any $g \in G$ we have that
\[
T(A \circ B)(g) = T(A)(B(g)) \cdot T(B)(g).
\]

If $G$ is abelian, or more generally $H \subset Z(G)$, the descended throw heights $T(A) : G/H \to G$ and $T(B) : G/H \to G$ satisfy
\[
T(A \circ B)(\ol{g}) = T(A)(\widetilde{B}(\ol{g})) \cdot T(B)(\ol{g}).
\]

\end{lemma}

\begin{proof}
The second equality will follow immediately from the first. To show the first equality holds, we compute
\begin{align*}
T(A)(B(g))\cdot T(B)(g) \cdot g &= T(A)(B(g))\cdot B(g) && \text{(by definition of $T(B)$)} \\
	&= A(B(g)),
\end{align*} 
by definition of $T(A)$ applied to the element $B(g)$. Thus, we have
\[
T(A)(B(g))\cdot T(B)(g) \cdot g = A \circ B(g),
\]
so 
\[
T(A)(B(g)) \cdot T(B)(g) = T_{A \circ B}(g),
\]
as desired.
\end{proof}

Let $\op{Perm}(G)$ denote the group of permutations of $G$ (no structure). Let $\op{Fun}(G, G)$ denote the set of functions to $G$, a group under pointwise multiplication. Then we have a right action of $\op{Perm}(G)$ on $\op{Fun}(G, G)$ given  by pre-composition, that is $T \cdot S = T \circ S$ for $T : G \to G$ and $S : G \to G$ a bijection. The first equality in the lemma shows that we get an injective homomorphism
\[
(T, \op{id}) : \op{Swap}_H(G) \hookrightarrow \op{Fun}(G, G) \rtimes \op{Perm}(G).
\]
That is, multiplication of the throw heights is twisted by pre-composition with $\op{Perm}(G)$. The second equality in the lemma shows that when $H \subset Z(G)$ we have an injective homomorphism
\[
(T, P) : \op{Swap}_H(G) \hookrightarrow \op{Fun}(G/H, G) \rtimes \op{Perm}(G/H).
\]
where the action in the codomain is also pre-composition. This is still injective because siteswaps are uniquely determined by their throw heights. Neither map is surjective, as we can not freely specify throw heights (some choices of throw heights yield collisions). The rest of this section amounts to obtaining and utilizing semi-direct product decompositions for $\op{Swap}_H(G)$ and relating them to the above maps.

The key ingredient will be showing that $P : \op{Swap}_H(G) \to \op{Perm}(G/H)$ splits, which we isolate out in the following lemma. 

\begin{lemma}
There exists a group homomorphism $\sigma : \op{Perm}(G/H) \to \op{Swap}_H(G)$ that is a section of $P$. If $K = \op{Ker}(P)$, then we have a semidirect product decomposition
\[
\op{Swap}_H(G) \cong K \rtimes \op{Perm}(G/H)
\]
\end{lemma}

\begin{proof}
The semidirect product decomposition is equivalent to the existence of a section, by the general theory of semidirect products. So we endeavor to construct $\sigma$.

To do this, let $\tau : G/H \to G$ be a section of $\pi : G \to G/H$. This is just a set function, not a homomorphism. Let $\widetilde{S} \in \op{Perm}(G/H)$. To define $\sigma(\widetilde{S}) = S$, we seek an $H$-equivariant bijection $S : G \to G$ such that the following diagram commutes
\[
\begin{tikzcd}
G \ar[r, "S"] \ar[d, "\pi"] & G \ar[d, "\pi"]\\
G/H \ar[r, "\widetilde{S}"] & G/H.
\end{tikzcd}
\]
In particular, note that for any $g \in G$ we have that 
\begin{align*}
\pi(g) = \pi(\tau \circ \pi(g))
\end{align*}
and so we get that $g[\tau \circ \pi(g)]^{-1} = h \in H$. Then by equivariance we'd be forced to define
\begin{align*}
S(g) = S(h \cdot \tau \circ \pi(g)) = h \cdot S(\tau \circ \pi(g)).
\end{align*}
Thus, $S$ is uniquely determined by its values on the image of $\tau$. Since $\tau$ is injective, we may define
\[
S(\tau(x)) = \tau \circ \widetilde{S}(x),
\]
and this determines $S$ uniquely, explicitly via 
\[
S(g) = (g[\tau \circ \pi(g)]^{-1}) \cdot \tau \circ \widetilde{S}(\pi(g)).
\]
If we can show that $S$ is $H$-equivariant, then since $\widetilde{S}$ is bijective, previous work will yield that $S$ is bijective. 

To do this, fix $g \in G$, and $h \in H$, and compute
\begin{align*}
S(hg) &= (hg[\tau \circ \pi(hg)]^{-1}) \cdot \tau \circ \widetilde{S}(\pi(hg))) \\
	&= (hg[\tau \circ \pi(g)]^{-1}) \cdot \tau \circ \widetilde{S}(\pi(g))) \\
	&= h S(g),
\end{align*}
as desired.

Thus, all that remains is to show that this resulting function $\sigma : \op{Perm}(G/H) \to \op{Swap}_H(G)$ respects composition, which is routine but a little finicky. To do this, let $A, B \in \op{Swap}_H(G)$, and let $g \in G$. Then we may uniquely write $g = h\tau(x)$ with $h \in H$ and $x \in G/H$. We compute
\begin{align*}
\sigma(A \circ B)(g) &= \sigma(A \circ B)(h\tau(x)) \\
	&= h \cdot \sigma(A \circ B)(\tau(x)) && \text{(equivariance of $\sigma(A \circ B)$)}\\
	&= h \cdot \tau \circ \widetilde{A \circ B}(x) && \text{(definition of $\sigma$)} \\
	&= h \cdot \tau \circ \widetilde{A} \circ \widetilde{B}(x) \\
	&= h \cdot \sigma(A)\Big(\tau \circ \widetilde{B}(x)\Big) && \text{(definition of $\sigma(A)$)}\\
	&= \sigma(A)\Big( h \cdot \tau \circ \widetilde{B}(x)\Big) && \text{(equivariance)} \\
	&= \sigma(A)\Big(\sigma(B)(h \cdot \tau(x))\Big) && \text{(definition of $\sigma(B)$)} \\
	&= \sigma(A) \circ \sigma(B)(h \cdot \tau(x)) \\
	&= \sigma(A) \circ \sigma(B)(g)
\end{align*} 
Thus $\sigma$ is a homomorphism, which by construction is a section to $P : \op{Swap}_H(G)$, and we get the desired semidirect product decomposition.

\end{proof}

Since we have the semidirect product structure, we are immediately motivated to investigate the structure of $K = \op{ker}(P)$.

\begin{theorem}
The throw heights function yields an inclusion
\[
T : K \hookrightarrow \op{Fun}(G, H)
\]
which is an anti-homomorphism. If $H \subset Z(G)$ then we get 
\[
T : K \xrightarrow{\sim} \op{Fun}(G/H, H)
\]
is an isomorphism, and
\[
\op{Swap}_H(G) \cong \op{Fun}(G/H, H) \rtimes \op{Perm}(G/H),
\]
where the action is pre-composition.
\end{theorem}

\begin{proof}
 By definition $K$ consists of all $H$-equivariant bijections $S : G \to G$ inducing the identity map on $\op{G/H}$. That is, we must have $\pi \circ S(g) = \pi(g)$ for all $g \in G$. That is, we have that $S(g) = hg$ for unique $h \in H$. That is, if $T(S) : G \to G$ are the throw heights, then we actually have $T(S) : G \to H$. The equivariance can again be described by saying that $T(S)(hg) = hT(S)(g)h^{-1}$ for any $g \in G$, $h \in H$. 

Reversing this, if we have some function $T : G \to H$ with $T(hg) = hT(g)h^{-1}$, and define $S : G \to G$ by $S(g) = T(g) \cdot g$, then our general theory yields that $S$ is $H$-equivariant, and by construction induces the identity map on $G/H$. Since the identity map is bijective, our general theory yields that $S$ is bijective, so these siteswaps are uniquely determined by their throw heights, and the throw heights may be chosen freely so long as they map to $H$ and are appropriately equivariant. 

For the group structure, let $A, B \in K$, and compute
\begin{align*}
T(A \circ B)(g) &= T(A)(B(g)) \cdot T(B)(g) \\
	&= T(A)\Big(T(B)(g) \cdot g\Big) \cdot T(B)(g) \\
	&= T(B)(g) \cdot  T(A)(g) \cdot [T(B)(g)]^{-1} \cdot T(B)(g) && (\text{as } T(B)(g) \in H) \\
	&= T(B)(g) \cdot T(A)(g)
%A \circ B(g) &= A(T(B)(g) \cdot g) \\
%	&= T(B)(g) \cdot A(g) && \text{(equivariance of $A$)} \\
%	&= T(B)(g) \cdot T(A)(g) \cdot g.
\end{align*}
This shows that $T(A \circ B) = T(B) \cdot T(A)$. That is, we have an anti-homomorphism induced by the inclusion $T: K \hookrightarrow \op{Fun}(G, H)$.

In the case where $H$ is Abelian, this is a homomorphism. If $H \subset Z(G)$, this image can be described more simply, as then we've seen that the throw heights are constant on the orbits of $H$, and as above any choice of throw heights will do. That is, when $H \subset Z(G)$ we get $T : K \xrightarrow{\sim} \op{Fun}(G/H, H)$, and the result holds.

\end{proof}


%\begin{theorem}
%Let $G$ be a group with $H \subset Z(G)$. Let $K = \ker(P)$ with $P : \op{Swap}_H(G) \to \op{Perm}(G/H)$. Let $T : \op{Swap}_H(G) \to \op{Fun}(G/H, G)$ be the function sending a siteswap to its throw height function. Then there is a subgroup $L \subset \op{Swap}_H(G)$ with $P|_L : L \to \op{Perm}(G/H)$ an isomorphism, yielding an isomorphism $K \rtimes L \cong \op{Swap}_H(G)$ induced by the inclusions fitting into a commutative diagram
%\[
%\begin{tikzcd}
%K \rtimes L \ar[r, "\cong"] \ar[d, "(T \text{, } P)"] \ar[d, "\cong"'] & \op{Swap}_H(G) \ar[d, hookrightarrow, "(T \text{, } P)"] \\
%\op{Fun}(G/H, H) \rtimes \op{Perm}(G/H) \ar[r, hookrightarrow] &  \op{Fun}(G/H, G) \rtimes \op{Perm}(G/H)
%\end{tikzcd}
%\]
%where the top and left arrows are isomorphisms, and the left arrow is an isomorphism on each term in the semidirect product.
%\end{theorem}
%
%\begin{proof}
%
%The desired subgroup $L$ is given by $L = \sigma(\op{Perm}(G/H))$, and by the general theory of semi-direct products we get an isomorphism $\varphi : K \rtimes L \xrightarrow{\sim} \op{Swap}_H(G)$ induced by the inclusions. So all that remains is to show that $(T, P) : K \rtimes L \to \op{Fun}(G/H, H) \rtimes \op{Perm}(G/H)$ has the presecribed codomain and is an isomorphism.
%
%We have already shown that the right vertical arrow $(T, P) : \op{Swap}(G) \to \op{Fun}(G/H, G) \rtimes \op{Perm}(G/H)$ is a homomorphism. The left arrow has the prescribed codomain by corollary \ref{abelian_equivariance_throw_heights}, and that lemma combined with the reasoning preceding this theorem statement shows that in fact $T: K \to \op{Fun}(G/H, H)$ is bijective. Thus, the left arrow is a bijection on each term in the semidirect product, and is a homomorphism by commutativity of the diagram and the bottom arrow being an injective homomorphism. Thus, the left arrow is an isomorphism, as desired.
% 
%\end{proof}
%Now we prove our result, that $T$ yields a homomorphism restricted to $K$, and that $T$ sends the action of conjugation of $K$ by $S \in \op{Swap}_H(G)$ to the action of $\widetilde{S}$ on $\op{Fun}(G/H, G)$ given by pre-composition. To do this, let $A, B \in K$, so they induce the identity permutation on $G/H$. Then the second equality in the lemma yields for any $x \in G/H$ that
%\[
%T(A \circ B)(x) = T(A)(\widetilde{B}(x)) + T(B)(x) = T(A)(x) + T(B)(x),
%\]
%the desired pointwise addition. 
%
%For the compatibility with the actions, let $S \in \op{Swap}_H(G)$ and let $A \in K$. Let $A' = S^{-1} \circ A \circ S$, and note that $A' \in K$ as kernels of homomorphisms are normal subgroups. Then we have $S \circ A' = A \circ S$. Thus, the second equality yields for any $x \in G/H$ that
%\begin{align*}
%T(A)(\widetilde{S}(x)) + T(S)(x) &= T(A \circ S)(x) \\
%	&= T(S \circ A')(x) \\
%	&= T(S)(\widetilde{A'}(x)) + T(A')(x) \\
%	&= T(S)(x) + T(A')(x) && \text{(as $A' \in K$, so $\widetilde{A'} = \id$)}.
%\end{align*}
%Cancelling the common $T(S)(x)$ term from both sides, we get that
%\[
%T(A)(\widetilde{S}(x)) = T(A')(x).
%\]
%Since $x \in G/H$ was arbitrary, we get that $T(A') = T(A) \circ \widetilde{S}$, the desired right action of $\op{Perm}(G/H)$. 
%
%We did it!

There are several pleasant consequences of this semidirect product decomposition, valuable in both discrete and continuous contexts, along the lines of the results for the \underline{\textcolor{blue}{\href{https://en.wikipedia.org/wiki/Affine_symmetric_group}{affine symmetric group}}}. The central idea is to either think of the spaces these siteswaps are acting on, or the space of the siteswaps themselves, as a kind of geometric object.

In a continuous context, let's say we have a nice group $G$, where nice can mean something like topological, smooth, algebraic, or analytic. Say we have a nice closed subgroup $H$ where $H \subset Z(G)$, and the quotient $G/H$ is also nice (for example if $G$ is Lie group, then $G/H$ exists as a Lie group when $H$ is closed, or $G/H$ is always a topological group when $G$ is). Then if $S : G \to G$ is an $H$-equivariant bijection, the equations $S(g) = T(g) \cdot g$ and $T(g) = S(g) \cdot g^{-1}$ show that the throw heights are nice if and only if the siteswap is. If $S$ is nice, it descends to a nice bijection $\widetilde{S} : G/H \to G/H$, and so the homomorphism $(T, P) : \op{Swap}_H(G) \to \op{Fun}(G/H, G) \rtimes \op{Perm}(G/H)$ sends the nice maps to pairs of nice maps. If there is a nice section $\tau : G/H \to G$ of $\pi$ (not necessarily a homomorphism), then the semidirect product decomposition $\op{Swap}_H(G) \cong \op{Fun}(G/H, H) \rtimes \op{Perm}(G/H)$ respects the nice maps.\footnote{For this, we'll need the property that a map $X \to H$ is nice if and only if it becomes nice after the inclusion $H \subset G$, in manifold theory this means $H$ is weakly embedded in $G$. This is needed so that the map $g \mapsto g \cdot [\tau \circ \pi(g)]^{-1} \in H$ is nice as a map into $H$, it is evidently nice as a map into $G$. Or you can reason via the Yoneda lemma if you interpret all points here in the ``functor of points" sense.} 

There are also some good results in the discrete context where $G/H$ is finite. Then we have $\op{Fun}(G/H, H) \cong H^n$ where $n = |G/H|$, which is a very geometric space, so we do geometry on the siteswaps themselves. \footnote{I imagine there's some interesting geometry to be done when these function spaces exist somehow in the category in question, though I have done no work in this direction.}

For instance, we have a natural homomorphism $\op{Sum} : \op{Fun}(G/H, G) \to G$ given by $T \mapsto \sum \limits_{x \in G/H} T(x)$\footnote{The finiteness is necessary here, as we are using an identification $G^{\times n} \cong G^{\oplus n}$.}. Furthermore, this sum is preserved under the action of $\op{Perm}(G/H)$. That is, if $\widetilde{S} \in \op{Perm}(G/H)$ we have that 
\[
\op{Sum}(T \circ \widetilde{S}) = \sum \limits_{x \in G/H} T(\widetilde{S}(x)),
\]
which, since $\widetilde{S}$ is bijective, is the same sum as that defining $\op{Sum}(T)$, just with the terms in a different order. Since addition is commutative by assumption, the sum is the same. Thus, we get a sum homomorphism 
\[
\op{Sum} : \op{Fun}(G/H, G) \rtimes \op{Perm}(G/H) \to G
\]
given by $\op{Sum}(T, \widetilde{S}) = \op{Sum}(T)$. We get a similar sum function for $\op{Fun}(G/H, H) \rtimes \op{Perm}(G/H) \to H$, fitting into a commutative diagram 
\[
\begin{tikzcd}
K \rtimes \op{Perm}(G/H) \ar[r, "\cong"] \ar[d, "(T \text{, } P)"] \ar[d, "\cong"'] & \op{Swap}_H(G) \ar[d, hookrightarrow] \\
\op{Fun}(G/H, H) \rtimes \op{Perm}(G/H) \ar[r, hookrightarrow] \ar[d, "\op{Sum}"] &  \op{Fun}(G/H, G) \rtimes \op{Perm}(G/H) \ar[d, "\op{Sum}"] \\
H \ar[r, hookrightarrow] & G.
\end{tikzcd}
\] 
This is another way of expressing part of the average theorem, that the sum of the throw heights in any $H$-equivariant siteswap actually lives in $H$. The other part comes from nothing that for siteswaps where every throw is in $H$, the action of $H$ on the orbits is quite simple. 

Appealing to the average theorem, we get the analogue of patterns with 0 balls as the kernel of the sum homomorphism, also interpretable as the hyperplane cut out by the equation $\sum \limits_{\ol{x} \in G/H} C_{\ol{x}} = 0$. Since the sum homomorphism sends $\op{Perm}(G/H)$ to zero, the kernel contains this subgroup. Thus, if we let $N$ denote the kernel of the sum homomorphism then we maintain the semidirect product decomposition 
\[
N \cong (N \cap \op{Fun}(G/H, H)) \rtimes \op{Perm}(G/H).
\]


Furthermore, since $G/H$ is a finite set, the structure of $\op{Perm}(G/H)$ is the structure of a finite symmetric group, which is incredibly well studied. For instance, we know that $\op{Perm}(G/H)$ is generated by transpositions, and if we pick elementary transpositions, we get the usual disjoint commutativity and braid relations.

To see geometrically how these permutations act on $\op{Fun}(G/H, G)$, first note that for any $\ol{x} \in \op{Fun}(G/H, G)$ there is a coordinate homomorphism 
\begin{align*}
C_{\ol{x}} :  \op{Fun}(G/H, G) &\to G \\
T &\mapsto T(\ol{x})
\end{align*}
 Under any chosen isomorphism $\op{Fun}(G/H, H) \cong H^n$, these are the standard coordinate functions. Then the action of $\op{Perm}(G/H)$ on this space is by permuting the coordinates. For the very similar group $\op{Fun}([n], \R) \rtimes S_n$, this is the natural action of $S_n$ on $\mathbb{R}^n$ by permuting the coordinates. And the kernel of the sum homomorphism is then the subspace of vectors with coordinates summing to zero, which for $n \geq 2$ yields the standard irreducible representation of $S_n$. 


We can get some nice geometry by focusing on transpositions. Fix $\ol{x}, \ol{y} \in G/H$. Let $\widetilde{S} \in \op{Perm}(G/H)$ be the transposition swapping $\ol{x}$ and $\ol{y}$. I claim that the action by $\widetilde{S}$ is a reflection about the hyperplane $C_{\ol{x}} - C_{\ol{y}} = 0$

Indeed, $\widetilde{S}^2 = \id$, so we just need to show that the fixed points of the action by $\widetilde{S}$ are exactly this hyperplane. Indeed, $\widetilde{S}$ exchanges the coordinates $C_{\ol{x}}$ and $C_{\ol{y}}$ and leaves the other coordinates unchanged, so we see that the fixed points are exactly those points where these two coordinates are equal, which is exactly the given hyperplane.

\begin{subsection}{Conjugacy Classes}

	Recall that conjugacy classes in $S_n$ are intimately connected with orbit decompositions. We attempt to tell the same story for siteswaps, since siteswaps are defined by their action on $G$, and as we've seen in the averae theorem the decomposition into orbits there can be very meaningful.
	
	Here we let $G$ be an abelian group, and $H$ a subgroup with $G/H$ finite. Let $P : \op{Swap}_H(G) \to \op{Perm}(G/H)$ be the projection homomorphism. Let $S \in \op{Swap}_H(G)$, and let $\wt{S} = P(S)$. The sets $\pi^{-1}(\ol{\cM})$ with $\ol{\cM}$ an orbit of $\wt{S}$ form a partition of $G$ which is preserved by the action of $S$. As such, for any orbit $\ol{\cM}$ of $\wt{S}$ we may define $S_{\ol{\cM}} : G \to G$ by
	\[
		S_{\ol{\cM}}(x) = \begin{cases}
			S(x) & \text{if } x \in \pi^{-1}(\ol{\cM}) \\
			x & \text{otherwise.}
		\end{cases}
	\]
	These are $H$-equivariant siteswaps which commute with each other and satisfy
	\[
		S = \prod \limits_{\ol{\cM}} S_{\ol{\cM}}.
	\]
	Furthermore, we have that $P(S_{\ol{\cM}})$ is a single cycle on $G/H$, acting on $\ol{\cM}$.
	Now let $A \in \op{Swap}_H(G)$ be arbitrary. Then we have
	\[
		A^{-1}SA = \prod \limits_{\ol{\cM}} A^{-1}S_{\ol{\cM}}A
	\]
	Since conjugation is a homomorphism, it preserves commutativity of elements. Furthermore, we have that
	\[
		\wt{B} := P(A^{-1}S_{\ol{\cM}}A) = P(A)^{-1}\wt{S}_{\ol{\cM}}P(A).
	\]
	Then we have that $\wt{B}$ is a single cycle of the same order as $\wt{S}_{\ol{\cM}}$ on $\ol{\cO} := P(A)(\ol{\cM})$. Furthermore we have that
	\begin{align*}
		\op{Sum}(A^{-1}S_{\ol{\cM}}A) &= \op{Sum}(A)^{-1}\op{Sum}(S_{\ol{\cM}}) \op{Sum}(A) \\
			&= \op{Sum}(S_{\ol{\cM}})
	\end{align*}
	since the sum homomorphism has values in an abelian group.

	This shows that conjugate siteswaps induce conjugate permutations, in such a way that map on orbits given by conjugation preserves throw sums. It turns out that the reverse is true as well. 

	\begin{theorem}
		Let $A, B \in \op{Swap}_H(G)$, and let $P : \op{Swap}_H(G) \to \op{Perm}(G/H)$ send a siteswap to its associated permutation on $G/H$. Then $A$ and $B$ are conjugate in $\op{Swap}_H(G)$ if and only if there is some $\gamma \in \op{Perm}(G/H)$ with $P(A) = \gamma^{-1}P(B)\gamma$ where for any orbit $\ol{\cM}$ of $P(A)$, we have that $\ol{\cM}$ and $\gamma(\ol{\cM})$ have the same sum of throw heights.
		
		That is, two siteswaps are conjugate if and only if they induce conjugate permutations with the same number of balls in each orbit.
	\end{theorem}

	\begin{proof}
		The "only if" direction was proven above. Now we prove that reverse direction, namely that if $A, B \in \op{Swap}_H(G)$ are two siteswaps with $P(A)$ and $P(B)$ conjugate in such a way that the conjugation preserves the sum of throw heights on each orbit, then $A$ and $B$ are conjugate siteswaps. 

Indeed, suppose $P(A) = \gamma^{-1}P(B)\gamma$ for some $\gamma \in \op{Perm}(G/H)$. We will build an element $C \in \op{Swap}_H(G)$ such that $A = C^{-1}BC$. 

Throughout, let 
\begin{align*}
	\wt{A} &= P(A) \\
	\wt{B} &= P(B)
\end{align*}

To motivate what follows, suppose we had such a $C$, and fix an $x_0 \in G$. Then since conjugation is a homomorphism, we would have for any integer $k$ that
\[
	C \circ A^k(x_0) = B^k \circ C(x_0)
\]
So we will build a $C$ satisfying this property over a choice of $x_0$ ranging over the orbits of $A$.

To do this, let's first define sections $\sigma, \tau: G/H \to G$ suited to this situation. For each orbit $\ol{\cM}$ of $A$, fix an element $\ol{x}_{\ol{\cM}} \in \ol{\cM}$, and define $\sigma(\ol{x}_{\ol{\cM}} ) = x_{\ol{\cM}}$ arbitrarily. Note that for $0 \leq r < |\ol{\cM}|$ the elements $\wt{A}^r(\ol{x}_{\ol{\cM}})$ are all distinct and exhaust $\ol{\cM}$, so we may define $\sigma$ on $\ol{\cM}$ so that
\[
	\sigma \circ \wt{A}^r(\ol{x}_{\ol{\cM}}) := A^r \circ \sigma(\ol{x}_{\ol{\cM}}).
\] 

Note that this is a section, as by definition of $\wt{A}$ we have
\begin{align*}
	\pi \circ A^r \circ \sigma(\ol{x}_{\ol{\cM}}) &= \wt{A}^r \circ \pi \circ \sigma(\ol{x}_{\ol{\cM}}) \\
	&= \wt{A}^r(\ol{x}_{\ol{\cM}}).
\end{align*}
% Recall from the main body of the text that
% \begin{align*}
% 	A^{|\ol{\cM}|} \circ \sigma(\ol{x}_{\ol{\cM}}) = \sigma(\ol{x}_{\ol{\cM}}) + \op{Sum}(\ol{\cM}).
% 	% A \circ \sigma \circ \wt{A}^{-1}(\ol{x}_{\ol{\cM}}) &= A \circ \sigma \circ \wt{A}^{|\ol{\cM}| - 1}(\ol{x}_{\ol{\cM}}) \\
% 	% &= A \circ A^{|\ol{\cM}| - 1} \circ \sigma(\ol{x}_{\ol{\cM}})  \\
% 	% &= A^{|\ol{\cM}|} \circ \sigma (\ol{x}_{\ol{\cM}}) \\
% 	% &= \sigma(\ol{x}_{\ol{\cM}}) + \op{Sum}(\ol{\cM}).
% \end{align*}

% Iterating this, if $k = |\ol{\cM}|\cdot q + r$ with $0 \leq r < |\ol{\cM}|$ then we have that 
% \begin{align*}
% 	A^k \circ \sigma(\ol{x}_{\ol{\cM}}) &= \sigma \circ \wt{A}^k(\ol{x}_{\ol{\cM}}) + q \op{Sum}(\ol{\cM}) \\
% 	&= \sigma \circ \wt{A}^r(\ol{x}_{\ol{\cM}}) + q \op{Sum}(\ol{\cM})
% \end{align*}

Define $\tau$ similarly, picking $\tau(\gamma(\ol{x}_{\ol{\cM}}))$ arbitrarily and defining $\tau$ based on that element, so that $\tau$ commutes with $B$ appropriately on the orbit $\ol{\cO} := \gamma(\ol{\cM}).$ The fact that $\gamma$ conjugates $B$ to $A$ yields that as $\ol{\cM}$ ranges over the orbits of $A$, $\gamma(\ol{\cM})$ ranges over the orbits of $B$, and so $\tau$ is well defined. 

With these sections in play, we define $C : G \to G$ to be the unique $H$-equivariant map satisfying
\[
	C \circ \sigma = \tau \circ \gamma.
\]
The same methods as in the main body of the paper show that $C$ is a well-defined bijection with $P(C) = \gamma$. \footnote{If you like, you can first define $D$ so that $D \circ \sigma = \sigma \circ \gamma$ directly as in the paper, and then note that $\tau = R \circ \sigma$ for some $H$-invariant $R$ inducing the identity permutation on $G/H$ as $\tau(\ol{x}) = \sigma(\ol{x}) + h $ for some $h \in H$ depending on $\ol{x}$. Then we can define $C = R \circ D$. } I claim that in fact $A = C^{-1}BC$. We will verify this separately on each set $\pi^{-1}(\ol{\cM})$, as $G$ is the disjoint union of these preimages. 

For each such preimage, note that by $H$-equivariance, it suffices to show that $C \circ A = B \circ C$ on $\sigma(\ol{\cM})$, as every other element in $\pi^{-1}(\ol{\cM})$ is of the form $\sigma(x) + h$ for some $h \in H$ and $x \in \ol{\cM}$.

To do this, first let $0 \leq r < |\ol{\cM}| - 1$ and compute 
\begin{align*}
	C \circ A \circ \sigma \circ \wt{A}^r(\ol{x}_{\ol{\cM}}) &= C \circ A \circ A^r \circ \sigma(\ol{x}_{\ol{\cM}}) \\
		&= C \circ A^{r + 1} \circ \sigma (\ol{x}_{\ol{\cM}}) \\
		&= C \circ \sigma \circ \wt{A}^{r + 1}(\ol{x}_{\ol{\cM}}) && \text{(as $r + 1 < |\ol{\cM}|$)} \\
		&= \tau \circ \gamma \circ \wt{A}^{r + 1}(\ol{x}_{\ol{\cM}}) \\
		&= \tau \circ \wt{B}^{r + 1} \circ \gamma(\ol{x}_{\ol{\cM}}) && (\text{as } \wt{A} = \gamma^{-1}\wt{B}\gamma) \\
		&=\tau \circ \wt{B} \circ \wt{B}^r \circ \gamma(\ol{x}_{\ol{\cM}}) \\
		&= B \circ \tau \circ \wt{B}^r \circ \gamma(\ol{x}_{\ol{\cM}}) && (\text{as } r + 1 < |\ol{\cO}|)\\
		&= B \circ \tau \circ \gamma \circ \wt{A}^r(\ol{x}_{\ol{\cM}}) && (\text{as } A = \gamma^{-1}B\gamma) \\
		&= B \circ C \circ \sigma \circ \wt{A}^r(\ol{x}_{\ol{\cM}}).
\end{align*}

Thus, $C \circ A$ and $B \circ C$ agree on all these elements.The only element remaining in $\ol{\cM}$ is $\wt{A}^{|\ol{\cM}| - 1}(\ol{x}_{\ol{\cM}})$. For this, we compute

\begin{align*}
	C \circ A \circ \sigma \circ \wt{A}^{|\ol{\cM}| - 1}(\ol{x}_{\ol{\cM}}) &= C \circ A \circ A^{|\ol{\cM}| - 1} \circ \sigma(\ol{x}_{\ol{\cM}}) \\
		&= C \circ A^{|\ol{\cM}|} \circ \sigma (\ol{x}_{\ol{\cM}}) \\
		&= C  \Big( \sigma (\ol{x}_{\ol{\cM}}) + \op{Sum}(\ol{\cM}) \Big) \\
		&= C \circ \sigma (\ol{x}_{\ol{\cM}}) + \op{Sum}(\ol{\cM}) && (H-\text{equivariance})\\
		&= \tau \circ \gamma(\ol{x}_{\ol{\cM}}) + \op{Sum}(\ol{\cM})
\end{align*}
Similarly, we compute
\begin{align*}
	B \circ C \circ \sigma \circ \wt{A}^{|\ol{\cM}| - 1}(\ol{x}_{\ol{\cM}}) &=  B \circ \tau \circ \gamma \circ \wt{A}^{|\ol{\cM}| - 1}(\ol{x}_{\ol{\cM}}) \\
	&= B \circ \tau \circ \wt{B}^{|\ol{\cM}| - 1} \circ \gamma (\ol{x}_{\ol{\cM}}) \\
	&= B \circ B^{|\ol{\cM}| - 1} \circ \tau \circ \gamma (\ol{x}_{\ol{\cM}}) \\
	&= B^{|\ol{\cM}|} \circ \tau \circ \gamma(\ol{x}_{\ol{\cM}}) \\
	&= B^{|\ol{\cO}|} \circ \tau \circ \gamma(\ol{x}_{\ol{\cM}}) \\
	&= \tau \circ \gamma (\ol{x}_{\ol{\cM}}) + \op{Sum}(\ol{\cO}) \\
	&= \tau \circ \gamma (\ol{x}_{\ol{\cM}}) + \op{Sum}(\ol{\cM}) 
\end{align*}
where the last equality holds by assumption that $\gamma$ preserves the sums of throw heights on orbits. Thus, $C \circ A$ and $B \circ C$ also agree on this element, and so they agree everywhere on $\pi^{-1}(\ol{\cM})$, as desired.

\end{proof}

As a consequence of this result, we can obtain a set of representatives for conjugacy classes in $\op{Swap}_H(G)$, a kind of canonical form. Let $R$ be a set of representatives for conjugacy classes in $\op{Perm}(G/H)$. In the case $G = \Z$ and $H = n\Z$ so $G/H = \Z/n$ a common choice is to pick a partition of $n$, and fill the corresponding Young tableux with the integers from $1$ to $n$ in ascending order left to right, and then interpret each row in the tableux as a cycle in the permutation. 
Then any section $f : \op{Perm}(G/H) \to \op{Swap}_H(G)$ as given in the main body of the paper yields siteswaps with throw heights summing to zero. Furthermore, for any $r \in R$ we have that $r = \prod \limits_{\ol{\cM}} r_{\ol{\cM}}$ as a product of disjoint cycles over the orbits of $r$. And we have $f(r) = \prod \limits_{\ol{\cM}} f(r_{\ol{\cM}})$. Let $\varphi : \op{Orb}(r) \to G/H$ be such that $\varphi(\ol{\cM}) \in \ol{\cM}$. For any function $g : \op{Orb}(r) \to H$ we can define a siteswap $S : G \to G$ with throw heights $T : G/H \to G$ given by 
\[
	T(x) = \begin{cases}
		g(\ol{\cM}) & \text{if } x = \varphi(\ol{\cM}) \\
		0 & \text{otherwise}
	\end{cases}
\]
Then we have that $f(r) \circ S$ induces the same permutation as $r$, but with throw sum function given by $g$. I claim that these form a set of conjugacy class representatives for $\op{Swap}_H(G)$.

Thus, if $A$ is an aribtrary siteswap, then we know that $P(A)$ is conjugate to exactly one $r \in R$. If we let $S$ be constructed as above with $g$ the throw heights function of $A$ then the results of this lemma apply to $A$ and $f(r) \circ S$ and so these siteswaps are conjugate, as desired.

\begin{subsection}{An Example}
Let's work through one of these groups in a manner similar to affine permutations. Let's let $G = \Z^2$, and let $H = n\Z \times m\Z$. Then we have that $G/H \cong \Z/n \times \Z/m$, and so $|G/H| = nm$. We may think of throws as vectors pointing from the throw's origin to the throw's destination. The action of $\op{Perm}(G/H)$ is then by permuting these vectors. 

Note that \[
\op{Fun}(G/H, H) \cong (n\Z \times m\Z)^{nm}.
\]
This space will be $2nm$-dimensional, so only directly visualizable when $n = m = 1$ (a moving image in 3D can get some sense of it when $n = 2, m = 1$). But then $G/H$ is the trivial group, as is $\op{Perm}(G/H)$, and so the group of siteswaps is just $\Z^2$ with the usual addition. So we wish to reduce the dimensions, and one natural way is by looking at the kernel of the sum map. That is, we consider the short exact sequence
\[
0 \to (\Z \times \Z)^{mn - 1} \xrightarrow{f} (\Z \times \Z)^{mn} \xrightarrow{\op{Sum}} \Z \times \Z \to 0.
\]
That is, if we let $\vec{x}_i \in \Z \times \Z$ for $i = 0, \dots, mn - 1$, then we have the map
\[
\op{Sum}(\vec{x}_0, \vec{x}_1, \dots, \vec{x}_{mn-1}) = \sum \limits_{i=0}^{mn - 1} \vec{x}_i.
\]
The kernel is parametrized by the function $f : (\Z \times \Z)^{mn - 1} \to (\Z \times \Z)^{mn}$ given by
\[
f(\vec{x}_1, \vec{x}_2, \dots, \vec{x}_{mn - 1}) = \left(-\sum \limits_{i=1}^{mn - 1} \vec{x}_i, \vec{x}_1, \vec{x}_2, \dots, \vec{x}_{mn - 1} \right)
\]
Any permutation of the $mn$ coordinates $[0, \dots, mn - 1]$ that leaves coordinate 0 fixed acts via the same permutation of the $mn - 1$ coordinates $[1, \dots, mn - 1]$. If we look at the transposition swapping coordinates 0 and 1, we compute
\begin{align*}
\op{Swap} \circ f(\vec{x}_1, \vec{x}_2, \dots, \vec{x}_{mn - 1}) &= \op{Swap}\left(-\sum \limits_{i=1}^{mn - 1} \vec{x}_i, \vec{x}_1, \vec{x}_2, \dots, \vec{x}_{mn - 1} \right) \\
	&= \left(\vec{x}_1, -\sum \limits_{i=1}^{mn - 1} \vec{x}_i, \vec{x}_2, \dots, \vec{x}_{mn - 1} \right) \\
	&= f \left( -\sum \limits_{i=1}^{mn - 1} \vec{x}_i, \vec{x}_2, \dots, \vec{x}_{mn - 1} \right).
\end{align*}
That is, this transposition on this parametrization of the zero-ball subspace sends the first coordinate to the negative sum of all the coordinates. In the case $nm = 2$, we get that $nm - 1 = 1$, and so there is only one coordinate, and this transposition acts by simply negating it (keeping in mind that by coordinate here I mean an element of $\Z \times \Z$). And this negation preserves the subspace $\Z \times 2\Z$, which gives the inclusion of semidirect products \[\op{Swap}_H(G) \cong \op{Fun}(\Z/2, \Z \times 2\Z) \rtimes S_2 \hookrightarrow \op{Fun}(\Z/2, \Z \times \Z) \rtimes S_2.\] 

\end{subsection}





\section{Multi-dimensional siteswaps}

We will now use these results to analyze siteswaps in multiple dimensions. That is, we fix an integer dimension $d \geq 2$, look for subgroups $H \subset \Z^d$ such that $\Z^d/H$ is finite, and study $H$-equivariant bijections $S : \Z^d \to \Z^d$. 

Let's first consider the case $G = \Z^2$ and $H = n\Z \times m\Z$, with $n, m \geq 1$. We will reduce other more complicated cases to the appropriate dimensional equivalent of this one. Then we have that $G/H \cong \Z/n \times \Z/m$. Let $S : \Z^2 \to \Z^2$ be an $H$-equivariant bijection, with throw heights $T : \Z^2 \to \Z^2$ which descend to $T : \Z/n \times \Z/m \to \Z \times \Z$. We can interpret $S$ as a synchronous pattern juggled by infinitely many hands in one of two ways. We can either think of the horizontal coordinate as indexing time, and the vertical coordinate as indexing hands, or we can think of the horizontal coordinate as indexing hands and the vertical coordinate as indexing time.

By our general theory, we know that 
\[
A = \sum \limits_{i = 0}^{n-1} \sum \limits_{j=0}^{m-1} T'(i, j) \in H.
\]
Since $H$ has the basis $(n, 0)$ and  $(0, m)$, we may uniquely write $A = p(n, 0) + q(0, m)$. Our first question to answer is what are the meanings of these coefficients? Given our interpretation above, and knowledge of the average theorem for standard siteswaps, we might expect these are the numbers of balls in the pattern where the horizontal or vertical coordinates are time, respectively. 

To make this precise, and since the proof is the same, we'll make this interpretation in the general context of a symmetry group that is a direct sum. That is, suppose $G$ is an abelian group and that we have some subgroup $H = N \oplus K$ for two subgroups $N, K \subset G$ with $G/H$ finite. Consider the following diagram
\[
\begin{tikzcd}
G\ar[d, "\pi_K"'] \ar[dr, "\pi_H"] \ar[r, "\pi_N"] & G/N \ar[d, "\pi_{KN}"] \\
G/K \ar[r, "\pi_{NK}"'] & G/H
\end{tikzcd}
\]
Let $S : G \to G$ be an $H$-equivariant bijection (so it's also $K$-equivariant). Let $\mathcal{O}$ be an orbit of $S$. Then the stabilizer of $\mathcal{O}$ under the action of $H$ is by the average theorem equal to $\op{Sum}(\pi_H(\ol{\mathcal{O}})) \in H$. Since $H = N \oplus K$ we have that $\op{Sum}(\pi_H(\ol{\mathcal{O}})) = n_0 + k_0$ for unique $n_0 \in N, k_0 \in K$. 


Let $S_N, S_K, S_H$ denote the descent of $S$ to $G/N$, $G/K$, and $G/H$, respectively. These are all bijections by our general theory. Since $K \subset H$, we get that $S_K$ is equivariant with respect to $\pi_K(H) \cong N$, and similarly $S_N$ is equivariant with respect to $\pi_N(H) \cong  K$. Let $\mathcal{O}_N, \mathcal{O}_K,$ and $\mathcal{O}_H$ denote the descent of $\mathcal{O}$ to these respective quotients. I claim that \begin{align*}
\op{Stab}(\mathcal{O}_K) = \pi_K(\op{Stab}(\mathcal{O})) = \langle \pi_K(n_0) \rangle \\
\op{Stab}(\mathcal{O}_N) = \pi_N(\op{Stab}(\mathcal{O})) = \langle \pi_N(k_0) \rangle.
\end{align*}
The second equality in each row is clear. For the first equality, suppose $\pi_K(n)$ stabilizes $\mathcal{O}_K$ for some $n \in N$. Thus we have that $\mathcal{O}_K + \pi_K(n) = \mathcal{O}_K$. Thus, there exists some $k \in K$ such that
\[
\mathcal{O} + n + k = \mathcal{O}.
\]
Since $H = N \oplus K$ and $\op{Stab}(\mathcal{O}) = \langle (n_0, k_0) \rangle$ we have that $n + k = m(n_0 + k_0)$ so that $n = mn_0$, giving one containment for the first equality. The other containment is similar, and similar for the quotient by $N$. Thus, we get that 
\begin{align*}
\pi_{NK}^{-1}(\mathcal{O}_H) \cong \frac{\pi_K(N)}{\langle \pi_K(n_0)\rangle } \cong \frac{N}{\langle n_0 \rangle} \\
\pi_{KN}^{-1}(\mathcal{O}_H) \cong \frac{\pi_N(K)}{\langle \pi_N(k_0)\rangle } \cong \frac{K}{\langle k_0\rangle }
\end{align*}
That is, the number of balls along the $N$ direction, modding out by $K$, is $N/n_0$, and the number of balls along the $K$ direction, modding out by $N$, is $K/k_0$, which is lovely!

Back in the context of $G = \Z^2$ and $H = m\Z \times n\Z$, we have that 
\begin{align*}
\frac{\Z^2}{m\Z \times 0} &\cong \Z/m \times \Z \\
\frac{\Z^2}{0 \times n\Z} &\cong \Z \times \Z/n
\end{align*}

Thus if we have a siteswap $S : \Z^2 \to \Z^2$ with symmetry group $m\Z \times n\Z$ and throw sum $p(m, 0) + q(0, n)$, then modding out by $m\Z \times 0$ yields a synchronous pattern juggled by $m$ hands throwing at the same time using $q$ balls. Similarly modding out by $0 \times n\Z$ yields a synchronous pattern juggled by $n$ hands throwing at the same time with $p$ balls. Beautiful!

%
%To do this, note that $S$ is also equivariant with respect to the subgroup $H' = 0 \times m\Z \subset H$, with $G/H' \cong \Z \times \Z/m$, and so we get a diagram
%\[
%\begin{tikzcd}
%\Z \times \Z \ar[r, "S"] \ar[d, "\pi"] & \Z \times \Z \ar[d, "\pi"] \\
%\Z \times \Z/m  \ar[r, "\widetilde{S}"] & \Z \times \Z/m
%\end{tikzcd}
%\]
%with $\widetilde{S}$ also a bijection. Furthermore, we can show that $\widetilde{S}$ is equivariant with respect to the group $n\Z \times 0$. To do this, note that 
%\begin{align*}
%\widetilde{S}(an + i, \ol{j}) &= \pi \circ S(an + i, j) \\
%	&= \pi \circ S\Big((an, 0) + (i, j) \Big) \\
%	&= \pi\Big( (an, 0) + S(i, j)\Big) \\
%	&= \pi(an, 0) + \pi \circ S(i, j) \\
%	&= (an, 0) + \widetilde{S}(i, \ol{j}).
%\end{align*}
%
%We may most readily interpret $\widetilde{S}$ as an $m$-handed synchronous siteswap of period $n$. Indeed if $\widetilde{S}(i, \ol{j}) = (k, \ol{\ell})$, then this says that the throw from the hand $\ol{j}$ made at beat $i$ lands in the hand $\ol{\ell}$ at beat $k$, and all hands throw and catch at the same time. 
%For jugglers accustomed to synchronous siteswap, this is not the usual convention where we add an extra beat in between throws and double all the throw values, these patterns have no empty beats. Then the sum of the throw heights equals $\# \text{balls} \cdot n$ (interpreting the number of balls correctly if we have negative throw heights), where here the throw height refers only to the difference in the first coordinate. That is, if $\pi_1 : \Z \times \Z \to \Z$ denotes projection onto the first coordinate, we have
%\begin{align*}
%\# \text{balls} \cdot n &= \pi_1 \left(\sum \limits_{i = 0}^{n-1} \sum \limits_{j=0}^{m-1} T(i, j) \right)\\
%	&= \pi_1(A) \\
%	&= \pi_1(p(n, 0) + q(0, m)) \\
%	&= pn.
%\end{align*}
%Thus, we have that $p$ is the number of balls in the $m$-handed pattern we get by interpreting the second coordinate mod $m$, and similarly $q$ is the number of balls in the $n$-handed pattern we get by interpreting the first coordinate mod $n$. A beautiful result!
%
%In fact, this also motivates a great interpretation of $S$. We can think of $S$ as a \textbf{synchronous pattern with infinitely many hands} in two ways. One way is to think of the horizontal coordinates as indexing the beats in time and the vertical coordinates as indexing the different hands. Then interpreting the hands mod $m$ gives a pattern with $p$ balls. Or we can think of the horizontal coordinates as indexing the different hands, and the vertical coordinates as indexing the beats in time. Then interpreting the hands mod $n$ gives a pattern with $q$ balls. Beautiful!


\newpage

\begin{subsection}{Visualizing an example pattern}

Before moving to higher dimensions and more complicated symmetry groups, we pause to work through an example of one of these patterns. Let's say we want to build a juggling pattern $S : \Z^2 \to \Z^2$ with symmetry group $\Z/3 \times \Z/2$. Then it suffices to construct a permutation $\widetilde{S} : \Z/3 \times \Z/2 \to \Z/3 \times \Z/2$, choose throw heights that yield this permutation, via the relation $\widetilde{S}(\ol{x}) = \pi(T'(\ol{x})) + \ol{x}$. Then defining the throw heights $T$ via $T(x) = T'(\ol{x})$ we get our full juggling pattern.

Here is one choice of permutation, indicated in blue, together with a choice of throw heights yielding that permutation, indicated in orange.

\begin{figure}[h]
\includegraphics[scale=0.26, center]{mult_dim_example_descended}
\caption{The coordinates for points are (0, 0) at the bottom left and (2, 1) at the top right.}
\end{figure}
So for example notice that $(2, 1) + (2, 1) = (4, 2) = (1, 0) + 1\cdot(3, 0) + 1 \cdot (0, 2)$, so that mod $H$ we have that adding $(2, 1)$ sends $(2, 1)$ to $(1, 0)$, as indicated in the diagram for $\widetilde{S}$. Then using these throw heights and equivariance, we get the following pattern. 

\begin{figure}[h]
\includegraphics[scale=0.2, center]{mult_dim_example}
\end{figure}

\newpage
The symmetry of this figure is that if you slide this picture two units vertically, or three units horizontally, you'll get the same picture. 

Adding the throw heights together gives $(6, 6) = 2 \cdot (3, 0) + 3 \cdot (0, 2)$. Thus, we expect that if we build a 2-handed pattern along the horizontal direction by interpreting the vertical coordinate mod $2$, then we should get a pattern of period 3 with two balls. Similarly if we build a 3-handed pattern along the vertical direction by interpreting the horizontal coordinate mod 3, we should get a pattern of period 2 with 3 balls.

Let's represent the 2-handed horizontal pattern. To represent this, we'll draw a golden band through the points with y-coordinate between 0 and 1, and draw the 2-handed pattern in this golden band. The key feature of this band is how it relates to the map $\pi_1 : \Z \times \Z \to \Z \times \Z/2$. The relationship is that for any point $(x, y) \in \Z^2$, there is a unique point $(x, y')$ with $y' \in \{0, 1\}$ and $\pi_1(x, y) = \pi_1(x, y')$. Thus, the golden band maps bijectively to $\Z \times \Z/2$. We draw another another golden band shifted two units up, through the points with y-coordinate between 2 and 3, we could also imagine golden bands at every vertical shift by a multiple of 2 units in either direction, but we'll just need these two bands for the picture. Then the new juggling pattern results from gluing these two bands together vertically. That is, any throw made in band A that lands in band B gets altered to land in the spot in band A corresponding to that same spot in band B (here it's the spot shifted down by two units, though in general it would be some even vertical shift). Thus, we get the following picture:

\begin{figure}[h]
\includegraphics[scale=0.23, center]{multi_dim_horizontal_band}
\end{figure}

The new 2-handed pattern is the pattern formed in the lower golden band by the dashed arrows. The dashed arrows are what happen to the initial blue arrows after gluing the upper band to the lower band. For example, the short vertical orange dashed arrows go vertically from the bottom to the top of the band because the initial throw is thrown at the bottom of the band and lands in the same horizontal position in the top of the other band. Or the purple dashed arrows result from the blue diagonal arrows, as those travel two units horizontally from the top of the lower band to the bottom of the higher band, so after gluing the bands together they go two units horizontally from the top of the band to the bottom of the band. 

Isolating out this pattern gives the picture

\begin{figure}[h]
\includegraphics[scale=0.23, center]{horizontal_band_isolated}
\caption{The two balls are color coded as either pink with solid lines, or blue with dashed lines.}
\end{figure}

Furthermore, we can see from the picture that this pattern does indeed have two balls.

To write out a siteswap for these patterns, we'll label the top hand A and the bottom hand B and use use the symbol $n_A$ to mean a throw that lands in hand $A$, $n$ beats later, and similar for $n_B$. We'll write a vertical pair $\begin{array}{c} x \\ y \end{array}$ to mean the throw values for hands A and B respectively. Collecting this into a table yields
\[
\begin{array}{c c c c c }
A &\vline&  1_A & 2_B & 2_B \\
B &\vline&  0_A & 1_A & 0_B
\end{array}
\]
You can collapse this to a multiplex pattern, where instead of 2 hands throwing one ball each per beat, one hand throws 2 balls per beat, written as $[10][21][20]$, which is more succinctly written as $1[12]2$ which is quite easy with passive 2s but a fun little pattern with active 2s. You can watch the synchronous version being juggled \underline{\textcolor{blue}{\href{https://jugglingedge.com/help/siteswapanimator.php?Pattern=(2\%2C0x)(4x\%2C2x)(4x\%2C0)}{here}}}. You might notice that there is a moment where one ball appears to disappear and instantly appear in the other hand. That's because of the throw $0_A$, which is suppose to be thrown at the same beat in the other hand, requiring instantaneous transportation. We can get around this by swapping the landing sites of the two throws made at this beat, obtaining 
\[
\begin{array}{c c c c c }
A &\vline&  0_A & 2_B & 2_B \\
B &\vline&  1_A & 1_A & 0_B
\end{array}
\]
which is eminently jugglable, as you can see \underline{\textcolor{blue}{\href{https://jugglingedge.com/help/siteswapanimator.php?Pattern=(0\%2C2x)(4x\%2C2x)(4x\%2C0)}{here}}}
The more standard synchronous siteswap for this is 
\[
\begin{array}{c c c c c }
A &\vline&  0 & 4x & 4x \\
B &\vline&  2x & 2x & 0
\end{array}
\]
We get this from the previous pattern by inserting an extra beat where no throws are made in between every beat, and doubling all the throw values to account for this extra beat, and use an $x$ to indicate crossing throws.
 
\newpage 
For the vertical three-handed pattern, we form vertical golden bands of width three, and glue them together horizontally. This yields the following picture

\begin{figure}[h]
\includegraphics[scale=0.23, center]{multi_dim_vertical_band}
\end{figure}

\clearpage
And here is the resulting 3-handed pattern decluttered from the rest.

\begin{figure}[h]
\includegraphics[scale=0.23, center]{vertical_band_isolated}
\caption{The three balls are color coded as solid blue lines, solid pink lines, and dashed black lines.}
\end{figure}

We can see from the picture that this is a pattern with three balls, as expected. There's one ball on the right half of the band, bouncing between the middle and right-most spots, and then following the paths of any two successive beats in the leftmost column yields the other two balls. 

The siteswap for this pattern, labeling the hands A, B, C from left to right, with time moving forward going up this table, is 
\[
\begin{array}{c c c}
A & B & C \\
\hline 3_A & 1_C & 0_C \\
0_B & 1_A & 1_B 
\end{array}
\]
which is impossible to juggle currently, as it is impossible to instantly throw a ball from one hand to a different hand, but swapping the $0_B$ and $1_A$ in the second line yields
\[
\begin{array}{c c c}
A & B & C \\
\hline 3_A & 1_C & 0_C \\
1_A & 0_B & 1_B 
\end{array}
\]
Here's \underline{\textcolor{blue}{\href{https://media.giphy.com/media/CU0Xn1m2CXLPVP5l1w/giphy.gif}{a gif}}} of the pattern animated using \underline{\textcolor{blue}{\href{http://westerboer.net/w/?page_id=151}{JoePass}}} (the clubs here are all thrown with one more spin than you'd need, but I added that in so the animator would actually throw the clubs that are thrown in the same hand on the next beat, as opposed to just holding them). You can also collapse this to a multiplex pattern [011][310] = [11][31], \underline{\textcolor{blue}{\href{https://jugglingedge.com/help/siteswapanimator.php?Pattern=\%5B31\%5D\%5B11\%5D}{animated here}}}, which is devilishly hard!

\end{subsection}

\newpage
\begin{subsection}{General patterns in multiple dimensions}

Now we wish to extend this reasoning to higher dimensions and less obviously diagonal symmetry groups. Let $H \subset \Z^d$ be a subgroup with $\Z^d$ finite. We first motivate our analysis by comparison with the case of 2 dimensions and symmetry group $n\Z \times m\Z$. What we want is set of $d$ independent directions $w_1, \dots, w_d$ generating $\Z^d$ such that we can interpret any $H$-invariant bijection $S$ as a synchronous pattern with infinitely many hands in $d$ different ways, choosing each of the directions $w_i$ as a time direction, and using coordinates along the other $d - 1$ directions to index hands. If we let $P_i : \Z^d \to \Z$ be projection onto the direction $w_i$, then we get a linear map $P : \Z^d \to \Z^d$ sending $w_i$ to $e_i$, the standard basis vectors for $\Z^d$ where $e_i$ has 1 in coordinate $i$ and 0 for the other coordinates. Choosing the $w_i$ be be independent directions generating $\Z^d$ means that $P$ is an isomorphism. 

Let's say that the period of the pattern we get along the direction $w_i$ is $\ell_i$. What this means is that $P_i(H) \cong \ell_i \Z$. Our best then hope for $H$ is then to have some diagram
\[
\begin{tikzcd}
H \ar[r, hookrightarrow] \ar[d, "P"'] \ar[d, "\sim"] & \Z^d \ar[d, "P"] \ar[d, "\sim"']  \\
\prod_{i=1}^d \ell_i \Z \ar[r, hookrightarrow] & \Z^d
\end{tikzcd}
\]
where the vertical arrows are isomorphisms. 

Picking a finite set of generators for $H$ (which we may do as $\Z$ is Noetherian), we get a diagram of group homomorphisms. 
\[
\begin{tikzcd}
H  \ar[r, hookrightarrow, "\iota"] & \Z^d \\
\Z^e \ar[u, twoheadrightarrow, "\varphi"] \ar[ur, "\psi"']
\end{tikzcd}
\]
with $\varphi$ surjective. The idea is to change generators for $H$, and change basis on $\Z^d$, so that $H$ becomes a diagonal subgroup, and we can reason as above, after keeping track of how the throw heights and juggling function change. If we pre and post compose with isomorphisms $Q : \Z^e \to \Z^e$ and $P : \Z^d \to \Z^d$, we get a diagram
\[
\begin{tikzcd}
H  \ar[r, hookrightarrow, "\iota"] & \Z^d \ar[r, "P"] \ar[r, "\sim"'] & \Z^d \\
\Z^e \ar[u, twoheadrightarrow, "\varphi"] \ar[ur, "\psi"'] \\
\Z^e \ar[u, "Q"] \ar[u, "\sim"'] \ar[uurr, "\lambda"']
\end{tikzcd}
\]
We first note that from commutativity of the diagram, the fact that $\varphi$ is surjective, and $P$ and $Q$ are isomorphisms, we have that $\Z^d/\text{im}(\lambda) \cong \Z^d/H$ ($H$ is the image of $\varphi \circ Q$, so applying $Q$ doesn't change the quotient, and then applying $P$ applies an isomorphism to the quotient, or you can argue via universal properties). The Smith normal form for integer matrices guarantees that $P$ and $Q$ may be chosen so that $\lambda$, when considered as a $d \times e$ matrix, has the diagonal form 

\[
\lambda = \begin{bmatrix}
\ell_1 & 0 & 0 & & \dots & & 0 \\
0  & \ell_2 & 0 & & \dots & & 0 \\
\vdots  &  & \ddots  & & &  &\vdots \\
  &  &  &  \ell_r & \\
  &  &  &  & 0 \\
  &  &  &  &  & \ddots \\
 0 & &  & \dots & & & 0
\end{bmatrix},
\]
with $\ell_i$ dividing $\ell_{i+1}$,
or written in block form 
\[
\lambda = \begin{bmatrix}
 L & 0_{r \times e - r} \\
 0_{d - r \times r} & 0_{d-r \times e-r}
\end{bmatrix}
\]
with $L$ diagonal. First note that if say row $i$ of $\lambda$ equals zero, then that means that the infinite subgroup $K$ spanned by $(0, \dots, 0, 1, 0, \dots, 0)$ with a 1 in the $i$th position intersects $\im(\lambda)$ in only the zero vector, and so $K$ maps isomorphically to $\Z^d/\im(\lambda)$, contradicting that this quotient is a finite set. Thus, we must have no zero rows, so $r = d$, and $\lambda$ has the block form 
\[
\lambda = \begin{bmatrix}L & 0_{d \times {e - d}} \end{bmatrix}
\]
and in particular we have that $e \geq d$. If any of the diagonal entries of $D$ are zero, then we get a zero row in the matrix, yielding the same contradiction. So $D$ is a diagonal matrix with all its diagonal entries nonzero. Consider $\Z^{d} \subset \Z^e$ as the vectors with last $e - d$ coordinates all zero. Let $Q'$ be the restriction of $Q$ to this subset. This block form shows that the image of $\lambda' = \lambda \circ Q'$ equals the image of $\lambda$ (as the last $e - d$ basis elements all map to zero), which is $P(H)$. Furthermore, since $\lambda$ when restricted to $\Z^d$ is a diagonal matrix with all nonzero entries, we get that this restriction of $\lambda$ is actually injective. 

Thus, we have a diagram
\[
\begin{tikzcd}
H  \ar[r, hookrightarrow, "\iota"] & \Z^d \ar[r, "P"] \ar[r, "\sim"'] & \Z^d \\
\Z^d \ar[u, twoheadrightarrow, "\varphi'"] \ar[ur, "\psi"'] \ar[urr, bend right, "\lambda'"']
\end{tikzcd}
\]
with $\lambda'$ injective. Then we get that $\varphi'$ is injective. Indeed, if $\varphi'(x) = 0$ then we get that 
\begin{align*}
0 &= P \circ \iota \circ \varphi'(x) \\
	&= \lambda'(x),
\end{align*}
and since $\lambda'$ is injective, we get that $x = 0$. Thus, $\varphi'$ is actually an isomorphism. 

We will now focus our attention on just the first triangle in the diagram
\[
\begin{tikzcd}
H  \ar[r, hookrightarrow, "\iota"] & \Z^d \\
\Z^d \ar[u, "\varphi'"] \ar[u, "\cong"'] \ar[ur, "\psi"']
\end{tikzcd}
\]
This yields an internal direct sum $H = \bigoplus \limits_{i=1}^d \langle \varphi'(e_i) \rangle$, with $\{e_i\}$ the standard basis vectors on $\Z^d$. Let $v_i = \varphi'(e_i)$, and let $\pi_i : Z^d \to \langle v_i \rangle$ denote the $i$th projection homomorphism. Let $\mathcal{O} \subset \Z^d$ be an orbit of $S$, let $\mathcal{O}_i$ denote the descent of $\mathcal{O}$ to $\langle v_i \rangle$, and let $\ol{\mathcal{O}}$ denote the descent of $\mathcal{O}$ to $\Z^d/H$. 

Our general theory yields that $\op{Sum}(\ol{\mathcal{O}}) = \op{Stab}(\mathcal{O})$, and we have that $\pi_i(\op{Sum}(\ol{\mathcal{O}})) = \op{Stab}(\mathcal{O}_i)$. By the direct sum decomposition, we may write 
\[
\op{Sum}(\ol{\mathcal{O}}) = \sum \limits_{j=1}^d b_j v_j 
\]
for unique integers $b_j$. Applying the projection map $\pi_i$ yields $b_i v_i = \pi_i(\op{Sum}(\ol{\mathcal{O}}))$. Let $S_i$ denote the descent of $S$ after modding out by all the other $v_j$, the equivalent of a one-dimensional regular siteswap. Then we have 
\[
\{\text{Orbits of } S_i \text{ over } \ol{\mathcal{O}} \} \cong \langle v_i \rangle/ \langle b_i v_i \rangle \cong \Z/b_i,
\]
and so
\[
|\{\text{Orbits of } S_i \text{ over } \ol{\mathcal{O}} \}| = |b_i|.
\]
That is, the sum of throw heights in an orbit is a vector consisting of elements in each principal direction of $H$, and the scale of that vector along a given direction is exactly the number of balls in the juggling pattern we get along that same direction that follow this orbit. Beautiful!
\end{subsection}
%So with the choice of generators for $H$ given by $\varphi'$, we investigate what happens to the juggling function and throw heights upon applying an isomorphism $P$. 
%
%In general, when applying a change of basis like $P$, we expect functions to change via a corresponding change of basis formula. Thus, if $S : \Z^d \to \Z^d$ is $H$-invariant, the right object to introduce is the function $\widehat{S} : \Z^d \to \Z^d$ making this diagram commute
%\[
%\begin{tikzcd}
%\Z^d \ar[r, "S"] \ar[d, "P"] & \Z^d \ar[d, "P"] \\
%\Z^d \ar[r, "\widehat{S}"] & \Z^d
%\end{tikzcd}
%\]
%so that $\widehat{S} = P \circ S \circ P^{-1}$.
%
%I claim that $S$ is $H$-equivariant if and only if $\widehat{S}$ is $P(H)$-equivariant. Indeed, suppose $S$ is $H$-equivariant. Let $y \in \Z^d$, $h \in H$, and let $x = P^{-1}(y)$ so that $y = P(x)$. Then we compute 
%\begin{align*}
%\widehat{S}(P(h) + y) &= \widehat{S}(P(h) + P(x)) \\
%	&= \widehat{S}(P(h + x)) && \text{(as $P$ is a homomorphism)} \\
%	&= \widehat{S} \circ P(h + x) \\
%	&= P \circ S(h + x) && \text{(commutativity of the diagram)} \\
%	&= P(h + S(x)) && \text{(as $S$ is $H$-equivariant)} \\
%	&= P(h) + P(S(x)) \\
%	&= P(h) + \widehat{S}(P(x)) \\
%	&= P(h) + \widehat{S}(y),
%\end{align*}
%the desired equivariance. The reverse implication comes from exchanging the roles of $S$ and $\widehat{S}$, and switching $P$ with $P^{-1}$. Or, alternately, if you like diagrams, this result is commutativity of the right face of following cube
%\[
%\begin{tikzcd}
%& H \times \Z^d \ar[rr, "(P \text{, } P)"] \ar[dl, "(\id \text{, } S)"'] \ar[dd] && P(H) \times \Z^d \ar[dl, "(\id \text{, } \widehat{S})"'] \ar[dd, "+"] \\
%H \times \Z^d \ar[rr, crossing over, "\qquad \quad (P \text{, } P)"] \ar[dd, "+"'] & & P(H) \times \Z^d  \\
%& \Z^d \ar[rr, "\qquad P"] \ar[dl, "S"] && \Z^d \ar[dl, "\widehat{S}"] \\
%\Z^d \ar[rr, "P"] & & \Z^d \ar[from=2-3, crossing over]
%\end{tikzcd}
%\]
%where all the vertical arrows are addition. The front and back faces commute because $P$ is a homomorphism. The top and bottom faces commute by definition of $\widehat{S}$, and the left face commutes by the assumption that $S$ is $H$-equivariant. The preceding algebra amounts to a diagram chase around the right face of this cube, using the fact that $(P, P)$ is surjective. If you are familiar with diagram chasing, feel free to use that technique yourself to prove that the right face of this cube commutes. 
%
%To see how throw heights change when moving from $S$ to $\widehat{S}$, fix $x \in \Z^d$, Then we compute
%\begin{align*}
%\widehat{S}(P(x)) &= P \circ S(x) \\
%	&= P(T(x) + x) && \text{(definition of throw heights)} \\
%	&= P \circ T(x) + P(x),
%\end{align*}
%as $P$ is a homomorphism. Thus, if $\widehat{T}$ are the throw heights for $\widehat{S}$, then we have the commutative diagram
%\[
%\begin{tikzcd}
%\Z^d \ar[r, "T"] \ar[d, "P"] & \Z^d \ar[d, "P"] \\
%\Z^d \ar[r, "\widehat{T}"] & \Z^d,
%\end{tikzcd}
%\]
%a lovely bit of harmony. 
%
%By construction, $P(H)$ has generators $\ell_ie_i$, where $e_i = (0, \dots, 1, \dots, 0)$, with a 1 in the $i$-th position, for $1 \leq i \leq d$. That is, $\widehat{S}$ has symmetry group $P(H) = \ell_1\Z \times \ell_2\Z \times \dots \times \ell_d\Z$, with quotient space $\Z^d/P(H) \cong \Z/\ell_1 \times \Z/\ell_2 \times \dots \times \Z/\ell_d$. 
%
%To relate this diagonal decomposition back to $H$, we first recall the diagram
%\[
%\begin{tikzcd}
%H  \ar[r, hookrightarrow, "\iota"] & \Z^d \ar[r, "P"] \ar[r, "\sim"'] & \Z^d \\
%\Z^d \ar[u, "\varphi'"] \ar[ur, "\psi'"'] \ar[urr, bend right, "\lambda'"']
%\end{tikzcd}
%\]
%with $\varphi'$ an isomorphism. Let $w_i = P^{-1}(e_i)$, and let $v_i = \psi'(e_i)$. Then commutativity of the diagram yields that 
%\[
%P(v_i) = \lambda'(e_i) = \ell_i e_i = \ell_i P(w_i), 
%\]
%so that 
%\[
%v_i = \ell_i w_i.
%\]
%These $w_i$ will be the principal directions along which we get multiplex patterns. For any vector $v \in \Z^d$, we will let $\langle v \rangle$ denote the subgroup generated by $v$. So in particular we get $\langle w_i \rangle / \langle v_i \rangle \cong \Z/\ell_i$. Furthermore, we get homomorphisms $\pi_i : \Z^d \to \langle w_i \rangle$ defined by $\pi_i(w_j) = 0 $ for $j \neq i$, and $\pi_i(w_i) = w_i$. These fit into a diagram
%\[
%\begin{tikzcd}
%\Z^d \ar[r, "P"] \ar[d, "\pi_i"] & \Z^d \ar[d, "\widehat{\pi}_i"] \\
%\langle w_i \rangle \ar[r, "\sim"] & \Z
%\end{tikzcd}
%\]
%where the bottom arrow sends $w_i$ to 1. Since $v_i = \ell_i w_i$, we get a diagram
%\[
%\begin{tikzcd}
%\langle w_i \rangle \ar[d] \ar[r, "\sim"] & \Z \ar[d] \\
%\langle w_i \rangle/ \langle v_i \rangle \ar[r, "\sim"] & \Z/\ell_i
%\end{tikzcd}
%\]
%with the horizontal arrows isomorphisms. 
%
%To get the synchronous pattern along the direction $w_i$, we mod out all the other directions $w_j$ by $v_j$, obtaining a diagram 
%\[
%\begin{tikzcd}
%\Z^d \ar[r, "P"] \ar[d] & \Z^d \ar[d] \\
%\langle w_i \rangle \times \prod \limits_{j \neq i} \langle w_j \rangle/ \langle v_j \rangle \ar[r, "\sim"'] \ar[r, "\gamma"] & \Z \times \prod \limits_{j \neq i} \Z/\ell_j
%\end{tikzcd}
%\]
%where the vertical arrows are the quotient maps corresponding to the subgroups $H' = \langle v_j \rangle_{j \neq i}$ and $P(H') = \langle e_j \rangle_{j \neq i}$. Then as before we descend the juggling functions to the quotient by these subgroups of their full symmetry groups. Thus, we obtain a commutative cube with all horizontal arrows isomorphisms (pictured with $i = 1$ for notational simplicity) \footnote{Please forgive my use of a tilde on top of a hat, I know it looks bad.}
%\[
%\begin{tikzcd}
%& \Z^d \ar[rr, "P"] \ar[dl, "S"'] \ar[dd] && \Z^d \ar[dl, "\widehat{S}"] \ar[dd] \\
%\Z^d \ar[rr, crossing over, "P"] \ar[dd] & & \Z^d  \\
%&  \langle w_1 \rangle \times \prod \limits_{j =2}^d \langle w_j \rangle/ \langle v_j \rangle  \ar[rr, "\gamma"] \ar[dl, "\widetilde{S}"'] && \Z \times \prod \limits_{j \neq i} \Z/\ell_j \ar[dl, "\widetilde{\widehat{S}}"] \\
%\langle w_1 \rangle \times \prod \limits_{j =2}^d \langle w_j \rangle/ \langle v_j \rangle \ar[rr, "\gamma"] & & \Z \times \prod \limits_{j=2}^d \Z/\ell_j \ar[from=2-3, crossing over]
%\end{tikzcd}
%\]
% The top face of the cube commutes by definition of $\widehat{S}$. The front and back faces of the cube commute by what we've just shown. The left and right faces commute as we've shown that we can descend juggling patterns to the quotient. If you have any experience with diagram chasing, I'd recommend proving to yourself that the bottom face commutes, using the fact that back left vertical arrow is surjective (as are all the vertical arrows), as that's more effective than reading a proof. But for completeness' sake, to see that the bottom face commutes, we use the notation for all of the vertical maps in the picture $x \mapsto \ol{x}$, and $\gamma$ is the same map for the back bottom edge. So let $\ol{x} \in \langle w_i \rangle \times \prod \limits_{j \neq i} \langle w_i \rangle/ \langle v_i \rangle$. Then we have 
%\begin{align*}
%\gamma \circ \widetilde{S}(\ol{x}) &= \gamma(\ol{S(x)}) && \text{(left face commutes)} \\
%	&= \ol{P \circ S(x)} && \text{(front face commutes)} \\
%	&= \ol{\widehat{S} \circ P(x)} && \text{(top face commutes)} \\
%	&= \widetilde{\widehat{S}}(\ol{P(x)}) && \text{(right face commutes)} \\
%	&= \widetilde{\widehat{S}} \circ \gamma(\ol{x}) && \text{(back face commutes)}.
%\end{align*}
%Since $\ol{x}$ was arbitrary, the bottom face commutes, and thus the whole diagram commutes. 
%
%All we will use from this commutativity is the following. The number of balls in $\widetilde{S}$, defined as the number of forward-moving infinite orbits $\widetilde{S}$ minus the number of backward-moving infinite orbits, is equal to the number of balls in $\widetilde{\widehat{S}}$. This occurs because the commutativity of the bottom face and $\gamma$ being an isomorphism ensures $\gamma$ sends the orbits of $\widetilde{S}$ to the orbits of $\widetilde{\widehat{S}}$ bijectively, and $\gamma$ preserves the orientation. 
%
%
%We need one more commutative diagram, which just says that projecting onto the first coordinate is the same as modding out the other coordinates then projecting onto the first coordinate. That is, we have the commutative diagram with horizontal arrows isomorphisms
%\[
%\begin{tikzcd}
% & \Z^d \ar[d] \ar[rr, "P"] \ar[ddl, bend right, "\pi_1"'] & & \Z^d \ar[d] \\
% & \langle w_1 \rangle \times \prod \limits_{j=2}^d \langle w_j \rangle/ \langle v_j \rangle \ar[dl, "\pi_1"] \ar[rr, "\gamma"'] & & \Z \times \prod \limits_{j=2}^d \Z/\ell_j \ar[dl, "\widehat{\pi}_1"] \\
% \langle w_1\rangle \ar[rr, "\gamma_1"] & & \Z \ar[from=1-4, bend right, crossing over, "\widehat{\pi}_1"']
%\end{tikzcd}
%\]
%
%The curved square from top to bottom commutes as $\gamma_1 \circ \pi_1$ sends $w_1$ to 1 and $w_j$ to 0 for $j > 1$. And $P$ sends $w_j$ to $e_j$ for all $j$, so $\widehat{\pi}_1(w_j)$ is 1 if $j = 1$, and 0 otherwise, the same result. The left and right faces (curved triangles) commute by entirely similar reasoning. We've already shown that the back face commutes. Then the bottom square commutes via a diagram chase using that the left vertical arrow on the back face is surjective.
%
%So now to get at the number of balls. Let $x_1, \dots, x_k$ be coset representatives for $\Z^d/H$ (that is, their images in $\Z^d/H$ yield a complete list of $\Z^d/H$ without repeats). Then since $P$ is an isomorphism, we get that $y_i = P(x_i)$ are coset representatives for $\Z^d/P(H)$. Let $A = \sum \limits_{i = 1}^k T(x_i)$, the sum of the throw heights. 
%%We compute using the commutativity of the front fact of the cube
%%\begin{align*}
%%\gamma \circ \ol{A} &= \ol{P(A)} \\
%%	&= \ol{P \left(\sum \limits_{i = 1}^k T(x_i) \right) } \\
%%	&= \sum \limits_{i = 1}^k \ol{ P \circ T(x_i)} \\
%%	&= \sum \limits_{i=1}^k \ol{\widehat{T} \circ P(x_i)}
%%\end{align*}
%Then using the commutative prism we get 
%\begin{align*}
%\gamma_1 \circ \pi_1(A) &= \widehat{\pi}_1 \circ P(A) \\
%	&= \widehat{\pi}_1(\ol{P(A)}) \\
%	&= \widehat{\pi}_1 \left(\ol{P \left(\sum \limits_{i = 1}^k T(x_i) \right) } \right) \\
%	&= \sum \limits_{i=1}^k \widehat{\pi}_1 \left(\ol{P \circ T(x_i)}\right) \\
%	&= \sum \limits_{i=1}^k \widehat{\pi}_1 \left(\ol{\widehat{T} \circ P(x_i)}\right) && \text{(recalling the relationship of the throw heights)} \\
%	&= \sum \limits_{i=1}^k \widehat{\pi}_1\left(\ol{\widehat{T}(y_i)}\right),
%\end{align*}
%which is the sum of the throw values of the synchronous pattern $\widetilde{\widehat{S}}$, which by the average theorem equals $\ell_1 \cdot \# \text{balls in } \widehat{S}$. By previous reasoning, this is the same number of balls as for $S$.
%
%Thus, we have that $\gamma_1 \circ \pi_1(A) = \ell_1 \cdot \# \text{balls}$. 
%
%However, since as we've shown previously, $A \in H$, we may uniquely write 
%\[
%A = \sum \limits_{i=1}^d b_i v_i = \sum \limits_{i=1}^d b_i \ell_i w_i.
%\]
%Thus, we have that $\pi_1(A) = b_1\ell_1 w_1$, so that $\gamma_1 \circ \pi_1(A) = b_1\ell_1$. Thus, we have
%\[
%b_1 \ell_1 = \ell_1 \cdot \# \text{balls},
%\]
%and so $b_1$ is the number of balls in the multiplex pattern we get along the $w_1$ direction. Similarly, $b_i$ is the number of balls in the synchronous pattern we get along the $w_i$ direction. 
%



\section{Miscellany}

We first remark on the composition of siteswaps with different periods. That is, if $H, K \subset G$ are normal subgroups, and $A, B : G \to G$ are equivariant with respect to $H$ and $K$, respectively, how much of the theory applies to $A \circ B$? First, we note that $A \circ B$ is equivariant with respect to $H \cap K$. The average theorem results applied when $G/H$ was finite. Luckily, we note that $H \cap K$ is the kernel of the homomorphism $f : G \to G/H \times G/K$, so if the latter two groups are finite, then $G/(H \cap K)$ is finite, as it's isomorphic to $\op{im}(f)$, which is necessarily finite as a subset of a finite set. So then if we just let $\op{Swap}(G)$ be the collection of all bijections $S : G \to G$ with $\op{Periods}(S)$ finite, then this set also forms a group under composition. To understand its structure, let $\mathcal{C}$ denote the poset category of normal subgroups $H \subset G$, where the morphisms are given by inclusion. The argument above shows that $\mathcal{C}^{op}$ is a filtered category, where where for any $H, K$ we have $H, K \to H \cap K$, and there are no parallel arrows to consider because it's a poset category. 

For any $H \in \mathcal{C}$ we get an inclusion $\iota_H : \op{Swap}_H(G) \hookrightarrow \op{Swap}(G)$, and by definition $\op{Swap}(G)$ is exactly the union of all the images of these maps. And if $H \subset K$ are normal subgroups with finite quotient, then then we get a commutative diagram
\[
\begin{tikzcd}
\op{Swap}_K(G) \ar[r] \ar[d] & \op{Swap}_H(G) \ar[dl] \\
\op{Swap}(G)
\end{tikzcd}
\]
The fact that $\op{Swap}(G)$ is the union over all these subgroups with finite quotient means that $\op{Swap}(G)$ is the filtered colimit over this diagram. We get a similar colimit if we remove the finite quotient restriction. We still get an inclusion homomorphism $\op{Swap}(G) \hookrightarrow \op{Fun}(G, G) \rtimes \op{Perm}(G)$ sending $S \mapsto (T(S), S)$, but I don't think we get a semidirect product decomposition for $\op{Swap}(G)$, as semidirect products decomposition of each $\op{Swap}_H(G)$ was non-canonical (it depended on a choice of section $G/H \to G$). Also the intersection over all subgroups of $\Z$ with finite quotient is just zero, so the permutation therm would likely be $\op{Perm}(\Z/0) = \op{Perm}(\Z)$ which already contains the desired group. 



Now for something completely different. Consider a direct product of abelian groups $G = N' \times K'$ and a subgroup $H \subset G$ satisfying $H = N \times K$ where $N = N' \cap H$ and $K = K' \cap H$. We are guaranteed to reduce down to this situation in the case $G = \Z^d$ by the Smith normal form for integer matrices. Then we have a diagram.
\[
\begin{tikzcd}
G\ar[d, "\pi_K"'] \ar[dr, "\pi_H"] \ar[r, "\pi_N"] & G/N \ar[d, "\pi_{KN}"] \\
G/K \ar[r, "\pi_{NK}"'] & G/H
\end{tikzcd}
\]
We'll let $H-\op{Equiv}(G)$ denote the set of all $H$-equivariant functions $G \to G$. If $S : G \to G$ is an $H$-equivariant function, then we get associated functions $S_K, S_N$ given by descending to $G/K$ and $G/N$, respectively, equivariant with respect to $N$ and $K$, respectively. It's a natural question to ask then to what extent do $S_K$ and $S_N$ determine $S$? It turns out that they fully determine $S$. Can they be chosen independently? It turns out that they can't, as they each descend to the same function $S_H$ on $G/H$. It turns out that this is the only restriction.

\begin{theorem}
Let $\gamma : H-\op{Equiv}(G) \to N-\op{Equiv}(G/K) \times K-\op{Equiv}(G/N)$ be the map $\gamma(S) = (S_K, S_N)$ described above. Then $\gamma$ is injective, and the image of $\gamma$ is exactly those elements $(A, B)$ which descend to the same function on $G/H$.
\end{theorem}

\begin{proof}
For injectivity, suppose $\gamma(A) = \gamma(B)$. Then for any $g \in G$ we have
\begin{align*}
\pi_K \circ A(g) &= A_K \circ \pi_k(g) \\
	&= B_K \circ \pi_K(g) \\
	&= \pi_K \circ B(g)
\end{align*}
and similarly we get
\[
\pi_N \circ A(g) = \pi_N \circ B(g).
\]
Thus, there exist $k \in K, n \in N$ such that
\[
A(g) = B(g) + k = B(g) + n.
\]
But then we get that $k = n$, and since $H = N \times K$ is a direct sum, we have that $k = n = 0$, and so $A(g) = B(g)$. Since $g$ was arbitrary, $A = B$. Note that this did not require the direct product decomposition on $G$, only the direct product decomposition on $H$.

For the statement about the image, we've already shown that pairs in the image have common descent to $G/H$, so we just need the reverse containment.

To do this, note that $G/K \cong N' \times K'/K$, and similar for $G/N$, and so we have the diagram
\[
\begin{tikzcd}
& & K' \\
& N' \times K' \ar[r, "\pi_N"] \ar[d, "\pi_K"] \ar[ur, "\pi_2"]  \ar[dl, "\pi_1"'] & N'/N \times K' \ar[d, "\pi_K"] \ar[u, "\pi_2"] \ar[dr, "\pi_1"] \\
N' & N' \times K'/K \ar[r, "\pi_N"] \ar[l, "\pi_1"] \ar[dr, "\pi_2"'] & N'/N \times K'/K \ar[r, "\pi_1"] \ar[d, "\pi_2"] & N'/N \\
 & & K'/K
\end{tikzcd}
\]

So now suppose we have 
\begin{align*}
A : N' \times K'/K \to N' \times K'/K \\
B : N'/N \times K' \to N'/N \times K'
\end{align*}
of the appropriate equivariance with common descent to $C : N'/N \times K'/K \to N'/N \times K'/K$. 
We wish to extend this to a function $S : N' \times K' \to N' \times K'$ descending to $A$ and $B$. By commutativity, we'd be forced to have
\begin{align*}
\pi_1 \circ S(x) &= \pi_1 \circ \pi_K \circ S(x) \\
	&= \pi_1 \circ A \circ \pi_K(x).
\end{align*}
Similarly, we'd be forced to have
\begin{align*}
\pi_2 \circ S(x) = \pi_2 \circ B \circ \pi_N(x).
\end{align*}
These two equalities define $S$ uniquely as a function, so all we need to do is check the appropriate equivariance, and that $S$ does in fact descend to $A$ and $B$.

To check that $S$ descends to $A$ and $B$, we extend the above diagram into three dimensions
\[
\begin{tikzcd}
 & N' \times K' \ar[dl, "S"'] \ar[rr, "\pi_N"] \ar[dd] & & \frac{N'}{N}  \times K' \ar[dl, "B"] \ar[dd, "\pi_K"] \\
 N' \times K \ar[rr, crossing over] \ar[dd, "\pi_K"] & & \frac{N'}{N} \times K'  \\
 & N' \times \frac{K'}{K} \ar[dl, "A"] \ar[rr] & & \frac{N'}{N} \times \frac{K'}{K} \ar[dl, "C"] \\
 N' \times \frac{K'}{K} \ar[rr, "\pi_N"] & & \frac{N'}{N} \times \frac{K'}{K} \ar[from=2-3, crossing over]
\end{tikzcd} 
\]
We are trying to show that the top and left faces of this cube commute. We know that the other four faces of this cube commute. Indeed the bottom and right faces commuting is the statement that $A$ and $B$ have common descent $C$. The front and back faces commute by basic group theory/previous reasoning. If say we are trying to show that $S$ descends to $A$ along $\pi_K$, then we are trying to show the left face commutes. Both composite arrows along this face are maps into a product, so they are the same arrow if and only if they become the same after composing with the projections onto the two factors. They become the same upon composition with projection onto $N'$ by construction, as do the arrows in the top face after projection onto $K'$, so the only question for the left face is projection onto $\frac{K'}{K}$. To analyze this, we extend the diagram with two more commuting faces.
\[
\begin{tikzcd}
 & N' \times K' \ar[dl, "S"'] \ar[rr, "\pi_N"] \ar[dd] & & \frac{N'}{N}  \times K' \ar[dl, "B"] \ar[dd] \\
 N' \times K \ar[rr, crossing over] \ar[dd, "\pi_K"] & & \frac{N'}{N} \times K' \ar[rr, crossing over] & & K' \ar[dddll, bend left] \\
 & N' \times \frac{K'}{K} \ar[dl, "A"] \ar[rr] & & \frac{N'}{N} \times \frac{K'}{K} \ar[dl, "C"] \\
 N' \times \frac{K'}{K} \ar[rr] \ar[drr] & & \frac{N'}{N} \times \frac{K'}{K} \ar[d, "\pi_2"] \ar[from=2-3, crossing over] \\
  & & \frac{K'}{K}
\end{tikzcd} 
\]

Showing that the left face commutes after projection onto $K'/K$ amounts to a diagram chase around this cube. It's probably best for you to do this diagram chase yourself, but for completeness here's a \textcolor{blue}{\underline{\href{https://youtu.be/pygdsX-8r_o}{youtube video}}} of me going through it (writing this out in symbols would be a total mess).

For the equivariance, let $(x, y) \in G = N' \times K'$, and let $n \in N, k \in K$. Then we compute
\begin{align*}
S(x + n, y + k) &= \Big(\pi_1 \circ A \circ \pi_K(x + n, y + k), \pi_2 \circ B \circ \pi_N(x + n, y + k) \Big) \\
	&= \Big(\pi_1 \circ A \circ \pi_K(x + n, y), \pi_2 \circ B \circ \pi_N(x, y + k) \Big) && \text{(definition of quotient maps)} \\
	&= \Big(\pi_1(A(x, y) + n), \pi_2(B(x, y) + k) \Big) && \text{(equivariance of $A$ and $B$)} \\
	&= \Big(\pi_1(A(x, y)) + n, \pi_2(B(x, y)) + k \Big) && \text{(each $\pi_i$ is a homomorphism)} \\
	&= \Big(\pi_1(A(x, y)), \pi_2(B(x, y)) \Big) + (n, k) \\
	&= S(x, y) + (n, k),
\end{align*}
as desired.

\end{proof}

\bibliography{multi_dimensional_bib.bib}{}
\bibliographystyle{plain}



\end{document}